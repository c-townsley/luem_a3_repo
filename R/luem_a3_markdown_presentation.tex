% Options for packages loaded elsewhere
\PassOptionsToPackage{unicode}{hyperref}
\PassOptionsToPackage{hyphens}{url}
%
\documentclass[
  ignorenonframetext,
]{beamer}
\usepackage{pgfpages}
\setbeamertemplate{caption}[numbered]
\setbeamertemplate{caption label separator}{: }
\setbeamercolor{caption name}{fg=normal text.fg}
\beamertemplatenavigationsymbolsempty
% Prevent slide breaks in the middle of a paragraph
\widowpenalties 1 10000
\raggedbottom
\setbeamertemplate{part page}{
  \centering
  \begin{beamercolorbox}[sep=16pt,center]{part title}
    \usebeamerfont{part title}\insertpart\par
  \end{beamercolorbox}
}
\setbeamertemplate{section page}{
  \centering
  \begin{beamercolorbox}[sep=12pt,center]{part title}
    \usebeamerfont{section title}\insertsection\par
  \end{beamercolorbox}
}
\setbeamertemplate{subsection page}{
  \centering
  \begin{beamercolorbox}[sep=8pt,center]{part title}
    \usebeamerfont{subsection title}\insertsubsection\par
  \end{beamercolorbox}
}
\AtBeginPart{
  \frame{\partpage}
}
\AtBeginSection{
  \ifbibliography
  \else
    \frame{\sectionpage}
  \fi
}
\AtBeginSubsection{
  \frame{\subsectionpage}
}
\usepackage{amsmath,amssymb}
\usepackage{lmodern}
\usepackage{iftex}
\ifPDFTeX
  \usepackage[T1]{fontenc}
  \usepackage[utf8]{inputenc}
  \usepackage{textcomp} % provide euro and other symbols
\else % if luatex or xetex
  \usepackage{unicode-math}
  \defaultfontfeatures{Scale=MatchLowercase}
  \defaultfontfeatures[\rmfamily]{Ligatures=TeX,Scale=1}
\fi
% Use upquote if available, for straight quotes in verbatim environments
\IfFileExists{upquote.sty}{\usepackage{upquote}}{}
\IfFileExists{microtype.sty}{% use microtype if available
  \usepackage[]{microtype}
  \UseMicrotypeSet[protrusion]{basicmath} % disable protrusion for tt fonts
}{}
\makeatletter
\@ifundefined{KOMAClassName}{% if non-KOMA class
  \IfFileExists{parskip.sty}{%
    \usepackage{parskip}
  }{% else
    \setlength{\parindent}{0pt}
    \setlength{\parskip}{6pt plus 2pt minus 1pt}}
}{% if KOMA class
  \KOMAoptions{parskip=half}}
\makeatother
\usepackage{xcolor}
\newif\ifbibliography
\usepackage{color}
\usepackage{fancyvrb}
\newcommand{\VerbBar}{|}
\newcommand{\VERB}{\Verb[commandchars=\\\{\}]}
\DefineVerbatimEnvironment{Highlighting}{Verbatim}{commandchars=\\\{\}}
% Add ',fontsize=\small' for more characters per line
\usepackage{framed}
\definecolor{shadecolor}{RGB}{248,248,248}
\newenvironment{Shaded}{\begin{snugshade}}{\end{snugshade}}
\newcommand{\AlertTok}[1]{\textcolor[rgb]{0.94,0.16,0.16}{#1}}
\newcommand{\AnnotationTok}[1]{\textcolor[rgb]{0.56,0.35,0.01}{\textbf{\textit{#1}}}}
\newcommand{\AttributeTok}[1]{\textcolor[rgb]{0.77,0.63,0.00}{#1}}
\newcommand{\BaseNTok}[1]{\textcolor[rgb]{0.00,0.00,0.81}{#1}}
\newcommand{\BuiltInTok}[1]{#1}
\newcommand{\CharTok}[1]{\textcolor[rgb]{0.31,0.60,0.02}{#1}}
\newcommand{\CommentTok}[1]{\textcolor[rgb]{0.56,0.35,0.01}{\textit{#1}}}
\newcommand{\CommentVarTok}[1]{\textcolor[rgb]{0.56,0.35,0.01}{\textbf{\textit{#1}}}}
\newcommand{\ConstantTok}[1]{\textcolor[rgb]{0.00,0.00,0.00}{#1}}
\newcommand{\ControlFlowTok}[1]{\textcolor[rgb]{0.13,0.29,0.53}{\textbf{#1}}}
\newcommand{\DataTypeTok}[1]{\textcolor[rgb]{0.13,0.29,0.53}{#1}}
\newcommand{\DecValTok}[1]{\textcolor[rgb]{0.00,0.00,0.81}{#1}}
\newcommand{\DocumentationTok}[1]{\textcolor[rgb]{0.56,0.35,0.01}{\textbf{\textit{#1}}}}
\newcommand{\ErrorTok}[1]{\textcolor[rgb]{0.64,0.00,0.00}{\textbf{#1}}}
\newcommand{\ExtensionTok}[1]{#1}
\newcommand{\FloatTok}[1]{\textcolor[rgb]{0.00,0.00,0.81}{#1}}
\newcommand{\FunctionTok}[1]{\textcolor[rgb]{0.00,0.00,0.00}{#1}}
\newcommand{\ImportTok}[1]{#1}
\newcommand{\InformationTok}[1]{\textcolor[rgb]{0.56,0.35,0.01}{\textbf{\textit{#1}}}}
\newcommand{\KeywordTok}[1]{\textcolor[rgb]{0.13,0.29,0.53}{\textbf{#1}}}
\newcommand{\NormalTok}[1]{#1}
\newcommand{\OperatorTok}[1]{\textcolor[rgb]{0.81,0.36,0.00}{\textbf{#1}}}
\newcommand{\OtherTok}[1]{\textcolor[rgb]{0.56,0.35,0.01}{#1}}
\newcommand{\PreprocessorTok}[1]{\textcolor[rgb]{0.56,0.35,0.01}{\textit{#1}}}
\newcommand{\RegionMarkerTok}[1]{#1}
\newcommand{\SpecialCharTok}[1]{\textcolor[rgb]{0.00,0.00,0.00}{#1}}
\newcommand{\SpecialStringTok}[1]{\textcolor[rgb]{0.31,0.60,0.02}{#1}}
\newcommand{\StringTok}[1]{\textcolor[rgb]{0.31,0.60,0.02}{#1}}
\newcommand{\VariableTok}[1]{\textcolor[rgb]{0.00,0.00,0.00}{#1}}
\newcommand{\VerbatimStringTok}[1]{\textcolor[rgb]{0.31,0.60,0.02}{#1}}
\newcommand{\WarningTok}[1]{\textcolor[rgb]{0.56,0.35,0.01}{\textbf{\textit{#1}}}}
\usepackage{graphicx}
\makeatletter
\def\maxwidth{\ifdim\Gin@nat@width>\linewidth\linewidth\else\Gin@nat@width\fi}
\def\maxheight{\ifdim\Gin@nat@height>\textheight\textheight\else\Gin@nat@height\fi}
\makeatother
% Scale images if necessary, so that they will not overflow the page
% margins by default, and it is still possible to overwrite the defaults
% using explicit options in \includegraphics[width, height, ...]{}
\setkeys{Gin}{width=\maxwidth,height=\maxheight,keepaspectratio}
% Set default figure placement to htbp
\makeatletter
\def\fps@figure{htbp}
\makeatother
\setlength{\emergencystretch}{3em} % prevent overfull lines
\providecommand{\tightlist}{%
  \setlength{\itemsep}{0pt}\setlength{\parskip}{0pt}}
\setcounter{secnumdepth}{-\maxdimen} % remove section numbering
\ifLuaTeX
  \usepackage{selnolig}  % disable illegal ligatures
\fi
\IfFileExists{bookmark.sty}{\usepackage{bookmark}}{\usepackage{hyperref}}
\IfFileExists{xurl.sty}{\usepackage{xurl}}{} % add URL line breaks if available
\urlstyle{same} % disable monospaced font for URLs
\hypersetup{
  pdftitle={Estimating a Flood Inundation Probability Map},
  pdfauthor={Riddhi Batra \& Charlie Townsley},
  hidelinks,
  pdfcreator={LaTeX via pandoc}}

\title{Estimating a Flood Inundation Probability Map}
\subtitle{CPLN 675 Assignment3}
\author{Riddhi Batra \& Charlie Townsley}
\date{2023-03-28}

\begin{document}
\frame{\titlepage}

\begin{frame}[fragile]{1. Introduction}
\protect\hypertarget{introduction}{}
The purpose of this project is to use existing flood inundation and
environmental data from one city to build a predictive model that
estimates flooding in a comparison city with similar characteristics.
Our model is trained and validated on the city of Calgary, in Alberta,
Canada. The city experienced record flooding in 2013 when heavy rainfall
met with melting snowpack in the Rocky Mountains. Nearly 80,000 people
were evacuated, and the flooding caused over \$5 billion in property
damage.\textsuperscript{1}

Predictive flood modeling has important implications for the field of
environmental planning. Floods cause most natural-disaster losses in the
United States and are responsible for an annual average of nearly \$8
billion in damage.\textsuperscript{2} Furthermore, as more people
develop property in flood-prone areas, inundation induced losses will
continue to increase. In the hands of government and community
organizations, the results from predictive models are vital for helping
cities plan, prepare, and respond to flooding.

With flood events increasing across the world, this project stresses the
importance of predictive, comparision-based modeling for cities that
face the potential of flooding, but have limited inundation history or
access to real-time flood monitoring. Such an approach can help cities
``on the edge'' make comparisons to better understand their own flood
exposure and consequently plan urban development, zoning and
preservation policy, and disaster-mitigation with a better sense of
preparedness.

\begin{block}{Why Denver?}
\protect\hypertarget{why-denver}{}
We chose Denver, Colorado as our comparison city for estimating flooding
because of key similarities. We observed that:

\begin{itemize}
\item
  Both cities straddle rivers and are adjacent to the Rocky Mountains
\item
  Both Calgary and Denver cover a large land area (319
  mi\textsuperscript{2} vs.~155 mi\textsuperscript{2})
\item
  They have similar levels of annual precipitation (16.7 in for Calgary
  and 8-15 in for Denver)
\item
  They are both at high elevations (3,400 ft for Calgary, 5,400 ft for
  Denver)
\item
  They both have flooded in the past
\end{itemize}
\end{block}

\begin{block}{Hypothesis}
\protect\hypertarget{hypothesis}{}
We began by thinking of which factors could influence and/or correlate
with flood inundation. Based on
\href{https://www.servirglobal.net/Portals/0/Documents/Articles/ISERV\%20Images/ISERV\%20Calgary_flood\%20V3.jpg}{aerial
images} from the Calgary flood of 2013, we hypothesized that the
probability of a grid cell in our model to flood is a function of
proximity to rivers, elevation, land cover permeability, and where water
has a natural tendency to flow and accumulation based on elevation and
slope.
\end{block}

\begin{block}{Project Structure}
\protect\hypertarget{project-structure}{}
This project follows the workflow below, which the following sections
describe in detail:

\begin{enumerate}
\item
  Setup

  \begin{itemize}
  \tightlist
  \item
    Gather open data from Calgary and Denver
  \item
    Process open data in ArcGIS to get our features of interest.
  \end{itemize}
\item
  Create Fishnets and Wrangle data

  \begin{itemize}
  \tightlist
  \item
    Create a fishnet shapefile in R for each city
  \item
    Use the Zonal Statistics as Table tool in ArcGIS to transform raster
    features to fishnet cell-level information
  \item
    Join zonal statistics tables with fishnet shapefiles in R
  \item
    Explore data in R
  \end{itemize}
\item
  Run Logistic Regressions

  \begin{itemize}
  \tightlist
  \item
    Train model on Calgary Data
  \item
    Refine model
  \item
    Validate model
  \end{itemize}
\item
  Interpretation
\item
  Map Predictions
\item
  Conclude with Closing Thoughts
\end{enumerate}
\end{block}

\begin{block}{1 Setup}
\protect\hypertarget{setup}{}
Load libraries, set working directories, write map and plot themes, and
call-in files.

\begin{Shaded}
\begin{Highlighting}[]
\NormalTok{mapTheme }\OtherTok{\textless{}{-}} \FunctionTok{theme}\NormalTok{(}\AttributeTok{plot.title =}\FunctionTok{element\_text}\NormalTok{(}\AttributeTok{size=}\DecValTok{12}\NormalTok{),}
                  \AttributeTok{plot.subtitle =} \FunctionTok{element\_text}\NormalTok{(}\AttributeTok{size=}\DecValTok{8}\NormalTok{),}
                  \AttributeTok{plot.caption =} \FunctionTok{element\_text}\NormalTok{(}\AttributeTok{size =} \DecValTok{6}\NormalTok{),}
                  \AttributeTok{axis.line=}\FunctionTok{element\_blank}\NormalTok{(),}
                  \AttributeTok{axis.text.x=}\FunctionTok{element\_blank}\NormalTok{(),}
                  \AttributeTok{axis.text.y=}\FunctionTok{element\_blank}\NormalTok{(),}
                  \AttributeTok{axis.ticks=}\FunctionTok{element\_blank}\NormalTok{(),}
                  \AttributeTok{axis.title.x=}\FunctionTok{element\_blank}\NormalTok{(),}
                  \AttributeTok{axis.title.y=}\FunctionTok{element\_blank}\NormalTok{(),}
                  \AttributeTok{panel.background=}\FunctionTok{element\_blank}\NormalTok{(),}
                  \AttributeTok{panel.border=}\FunctionTok{element\_blank}\NormalTok{(),}
                  \AttributeTok{panel.grid.major=}\FunctionTok{element\_line}\NormalTok{(}\AttributeTok{colour =} \StringTok{\textquotesingle{}transparent\textquotesingle{}}\NormalTok{),}
                  \AttributeTok{panel.grid.minor=}\FunctionTok{element\_blank}\NormalTok{(),}
                  \AttributeTok{legend.direction =} \StringTok{"vertical"}\NormalTok{, }
                  \AttributeTok{legend.position =} \StringTok{"right"}\NormalTok{,}
                  \AttributeTok{plot.margin =} \FunctionTok{margin}\NormalTok{(}\DecValTok{1}\NormalTok{, }\DecValTok{1}\NormalTok{, }\DecValTok{1}\NormalTok{, }\DecValTok{1}\NormalTok{, }\StringTok{\textquotesingle{}cm\textquotesingle{}}\NormalTok{),}
                  \AttributeTok{legend.key.height =} \FunctionTok{unit}\NormalTok{(}\DecValTok{1}\NormalTok{, }\StringTok{"cm"}\NormalTok{), }\AttributeTok{legend.key.width =} \FunctionTok{unit}\NormalTok{(}\FloatTok{0.2}\NormalTok{, }\StringTok{"cm"}\NormalTok{))}

\NormalTok{plotTheme }\OtherTok{\textless{}{-}} \FunctionTok{theme}\NormalTok{(}
  \AttributeTok{plot.title =}\FunctionTok{element\_text}\NormalTok{(}\AttributeTok{size=}\DecValTok{12}\NormalTok{),}
  \AttributeTok{plot.subtitle =} \FunctionTok{element\_text}\NormalTok{(}\AttributeTok{size=}\DecValTok{8}\NormalTok{),}
  \AttributeTok{plot.caption =} \FunctionTok{element\_text}\NormalTok{(}\AttributeTok{size =} \DecValTok{6}\NormalTok{),}
  \AttributeTok{axis.text.x =} \FunctionTok{element\_text}\NormalTok{(}\AttributeTok{size =} \DecValTok{10}\NormalTok{, }\AttributeTok{angle =} \DecValTok{45}\NormalTok{, }\AttributeTok{hjust =} \DecValTok{1}\NormalTok{),}
  \AttributeTok{axis.text.y =} \FunctionTok{element\_text}\NormalTok{(}\AttributeTok{size =} \DecValTok{10}\NormalTok{),}
  \AttributeTok{axis.title.y =} \FunctionTok{element\_text}\NormalTok{(}\AttributeTok{size =} \DecValTok{10}\NormalTok{),}
  \CommentTok{\# Set the entire chart region to blank}
  \AttributeTok{panel.background=}\FunctionTok{element\_blank}\NormalTok{(),}
  \AttributeTok{plot.background=}\FunctionTok{element\_blank}\NormalTok{(),}
  \CommentTok{\#panel.border=element\_rect(colour="\#F0F0F0"),}
  \CommentTok{\# Format the grid}
  \AttributeTok{panel.grid.major=}\FunctionTok{element\_line}\NormalTok{(}\AttributeTok{colour=}\StringTok{"\#D0D0D0"}\NormalTok{,}\AttributeTok{size=}\NormalTok{.}\DecValTok{75}\NormalTok{),}
  \AttributeTok{axis.ticks=}\FunctionTok{element\_blank}\NormalTok{())}

\CommentTok{\#color nerds}

\NormalTok{blues }\OtherTok{\textless{}{-}} \FunctionTok{c}\NormalTok{(}\StringTok{"\#CEEBF0"}\NormalTok{, }\StringTok{"\#A2D3D8"}\NormalTok{, }\StringTok{"\#73B8BF"}\NormalTok{, }\StringTok{"\#51A6AE"}\NormalTok{, }\StringTok{"\#468D94"}\NormalTok{, }\StringTok{"\#34696E"}\NormalTok{)}

\NormalTok{greens }\OtherTok{\textless{}{-}} \FunctionTok{c}\NormalTok{(}\StringTok{"\#D8E5CE"}\NormalTok{, }\StringTok{"\#C1D4B5"}\NormalTok{, }\StringTok{"\#A9BF99"}\NormalTok{, }\StringTok{"\#92AF7E"}\NormalTok{, }\StringTok{"\#81996F"}\NormalTok{, }\StringTok{"\#668053"}\NormalTok{)}

\NormalTok{yellows }\OtherTok{\textless{}{-}} \FunctionTok{c}\NormalTok{(}\StringTok{"\#F9E2B2"}\NormalTok{, }\StringTok{"\#EDC876"}\NormalTok{, }\StringTok{"\#EDBA46"}\NormalTok{, }\StringTok{"\#EDB025"}\NormalTok{, }\StringTok{"\#AC832D"}\NormalTok{, }\StringTok{"\#6E5321"}\NormalTok{)}

\NormalTok{greys }\OtherTok{\textless{}{-}} \FunctionTok{c}\NormalTok{(}\StringTok{"\#ECEBF1"}\NormalTok{, }\StringTok{"\#D3D3D9"}\NormalTok{, }\StringTok{"\#B0B0B3"}\NormalTok{, }\StringTok{"\#9C9C9E"}\NormalTok{, }\StringTok{"\#7F7F80"}\NormalTok{, }\StringTok{"\#656666"}\NormalTok{)}

\NormalTok{neutrals }\OtherTok{\textless{}{-}} \FunctionTok{c}\NormalTok{(}\StringTok{"\#FDF4E9"}\NormalTok{, }\StringTok{"\#FFEFDE"}\NormalTok{, }\StringTok{"\#F6D7B2"}\NormalTok{, }\StringTok{"\#D6B28B"}\NormalTok{, }\StringTok{"\#A58565"}\NormalTok{)}
\end{Highlighting}
\end{Shaded}

\begin{Shaded}
\begin{Highlighting}[]
\NormalTok{denver\_boundary }\OtherTok{\textless{}{-}}\FunctionTok{read\_sf}\NormalTok{(}\StringTok{"Denver/Processed/LUEM\_Asgn3\_Denver/NewData/denver\_bound/denver\_bound.shp"}\NormalTok{)}

\NormalTok{calgary\_boundary }\OtherTok{\textless{}{-}} \FunctionTok{read\_sf}\NormalTok{(}\StringTok{"Calgary/Raw/CALGIS\_CITYBOUND\_LIMIT/CALGIS\_CITYBOUND\_LIMIT.shp"}\NormalTok{)}
\end{Highlighting}
\end{Shaded}
\end{block}
\end{frame}

\begin{frame}[fragile]{2. Fishnets and Data Wrangling}
\protect\hypertarget{fishnets-and-data-wrangling}{}
\begin{block}{2.1 Create Fishnets}
\protect\hypertarget{create-fishnets}{}
Create a fishnet for Calgary to do processing in ArcGIS.

\begin{Shaded}
\begin{Highlighting}[]
\NormalTok{calgary\_fishnet }\OtherTok{\textless{}{-}} \FunctionTok{st\_make\_grid}\NormalTok{(calgary\_boundary,}
                        \AttributeTok{cellsize =} \FloatTok{402.336}\NormalTok{,}
                        \AttributeTok{square =} \ConstantTok{FALSE}\NormalTok{) }\SpecialCharTok{\%\textgreater{}\%} 
\NormalTok{  .[calgary\_boundary] }\SpecialCharTok{\%\textgreater{}\%} 
  \FunctionTok{st\_sf}\NormalTok{() }\SpecialCharTok{\%\textgreater{}\%} 
  \FunctionTok{mutate}\NormalTok{(}\AttributeTok{uniqueID =} \FunctionTok{rownames}\NormalTok{(.))}

\FunctionTok{ggplot}\NormalTok{()}\SpecialCharTok{+}
  \FunctionTok{geom\_sf}\NormalTok{(}\AttributeTok{data =}\NormalTok{ calgary\_fishnet,}
          \AttributeTok{fill =} \StringTok{"lightgrey"}\NormalTok{)}\SpecialCharTok{+}
  \FunctionTok{geom\_sf}\NormalTok{(}\AttributeTok{data =}\NormalTok{ calgary\_boundary, }
          \AttributeTok{color =} \StringTok{"\#6E5321"}\NormalTok{, }\AttributeTok{fill =} \StringTok{"transparent"}\NormalTok{) }\SpecialCharTok{+}
\NormalTok{  mapTheme}
\end{Highlighting}
\end{Shaded}

\includegraphics{luem_a3_markdown_presentation_files/figure-beamer/create_fishnets_calgary-1.pdf}

\begin{Shaded}
\begin{Highlighting}[]
\CommentTok{\#st\_write(calgary\_fishnet, "Calgary/Processed/calgary\_fishnet/calgary\_fishnet.shp", geometry = TRUE)}
\end{Highlighting}
\end{Shaded}

Create a fishnet for Denver to do processing in ArcGIS.

\begin{Shaded}
\begin{Highlighting}[]
\NormalTok{denver\_fishnet }\OtherTok{\textless{}{-}} \FunctionTok{st\_make\_grid}\NormalTok{(denver\_boundary,}
                        \AttributeTok{cellsize =} \DecValTok{1320}\NormalTok{,}
                        \AttributeTok{square =} \ConstantTok{FALSE}\NormalTok{) }\SpecialCharTok{\%\textgreater{}\%} 
\NormalTok{  .[denver\_boundary] }\SpecialCharTok{\%\textgreater{}\%} 
  \FunctionTok{st\_sf}\NormalTok{() }\SpecialCharTok{\%\textgreater{}\%} 
  \FunctionTok{mutate}\NormalTok{(}\AttributeTok{uniqueID =} \FunctionTok{rownames}\NormalTok{(.))}

\FunctionTok{ggplot}\NormalTok{()}\SpecialCharTok{+}
  \FunctionTok{geom\_sf}\NormalTok{(}\AttributeTok{data =}\NormalTok{ denver\_fishnet,}
          \AttributeTok{color=}\StringTok{"darkgrey"}\NormalTok{, }\AttributeTok{fill =} \StringTok{"lightgrey"}\NormalTok{) }\SpecialCharTok{+}
  \FunctionTok{geom\_sf}\NormalTok{(}\AttributeTok{data =}\NormalTok{ denver\_boundary, }
          \AttributeTok{color =} \StringTok{"\#6E5321"}\NormalTok{, }\AttributeTok{fill =} \StringTok{"transparent"}\NormalTok{) }\SpecialCharTok{+}
\NormalTok{  mapTheme}
\end{Highlighting}
\end{Shaded}

\includegraphics{luem_a3_markdown_presentation_files/figure-beamer/create_fishnets_denver-1.pdf}

\begin{Shaded}
\begin{Highlighting}[]
\CommentTok{\#st\_write(denver\_fishnet, "Denver/Processed/R exports/denver\_fishnet/denver\_fishnet.shp", geometry = TRUE)}
\end{Highlighting}
\end{Shaded}
\end{block}

\begin{block}{2.2 Feature Engineering: Arc to R}
\protect\hypertarget{feature-engineering-arc-to-r}{}
After creating fishnets for Calgary and Denver in R, we exported them as
shapefiles for processing in ArcGIS. We created tables of zonal
statistics from relevant raster data, by fishnet cell, in ArcGIS. Then,
we brought these tables back into R and joined each table with their
respective city's fishnet.

\begin{Shaded}
\begin{Highlighting}[]
\CommentTok{\#load in cleaned fishnet with partial cells removed}
\NormalTok{calgary\_fishnet }\OtherTok{\textless{}{-}}  \FunctionTok{read\_sf}\NormalTok{(}\StringTok{"Calgary/Processed/calgary\_fishnet\_nozeros/calgary\_fishnet\_nozeros.shp"}\NormalTok{)}

\CommentTok{\#load in engineered features for calgary and process}
\NormalTok{calg\_inund }\OtherTok{\textless{}{-}} \FunctionTok{read\_csv}\NormalTok{(}\StringTok{"Calgary/Processed/zonalstats\_tables/calg\_inundation\_sum.csv"}\NormalTok{) }\SpecialCharTok{\%\textgreater{}\%}
  \FunctionTok{rename}\NormalTok{(}\AttributeTok{inund\_sum =}\NormalTok{ SUM) }\SpecialCharTok{\%\textgreater{}\%} 
\NormalTok{  dplyr}\SpecialCharTok{::}\FunctionTok{select}\NormalTok{(uniqueID, inund\_sum) }\SpecialCharTok{\%\textgreater{}\%} 
  \FunctionTok{mutate}\NormalTok{(}\AttributeTok{inund\_sum =} \FunctionTok{ifelse}\NormalTok{(inund\_sum }\SpecialCharTok{\textgreater{}=} \DecValTok{16}\NormalTok{, }\DecValTok{1}\NormalTok{, }\DecValTok{0}\NormalTok{)) }\CommentTok{\#turn inundation sum values into binary (threshold = 16)}

\NormalTok{calg\_pervious }\OtherTok{\textless{}{-}} \FunctionTok{read.csv}\NormalTok{(}\StringTok{"Calgary/Processed/zonalstats\_tables/calg\_pervious\_mean.csv"}\NormalTok{) }\SpecialCharTok{\%\textgreater{}\%} 
\NormalTok{  dplyr}\SpecialCharTok{::}\FunctionTok{select}\NormalTok{(uniqueID, MEAN) }\SpecialCharTok{\%\textgreater{}\%} 
  \FunctionTok{rename}\NormalTok{(}\AttributeTok{pervious\_mean =}\NormalTok{ MEAN)}

\NormalTok{calg\_elevation }\OtherTok{\textless{}{-}} \FunctionTok{read.csv}\NormalTok{(}\StringTok{"Calgary/Processed/zonalstats\_tables/calg\_elevation\_mean.csv"}\NormalTok{) }\SpecialCharTok{\%\textgreater{}\%} 
\NormalTok{    dplyr}\SpecialCharTok{::}\FunctionTok{select}\NormalTok{(uniqueID, MEAN) }\SpecialCharTok{\%\textgreater{}\%} 
  \FunctionTok{rename}\NormalTok{(}\AttributeTok{elevation\_mean =}\NormalTok{ MEAN)}

\NormalTok{calg\_flowac }\OtherTok{\textless{}{-}} \FunctionTok{read.csv}\NormalTok{(}\StringTok{"Calgary/Processed/zonalstats\_tables/calg\_flowac\_mean.csv"}\NormalTok{) }\SpecialCharTok{\%\textgreater{}\%} 
\NormalTok{    dplyr}\SpecialCharTok{::}\FunctionTok{select}\NormalTok{(uniqueID, MEAN) }\SpecialCharTok{\%\textgreater{}\%} 
  \FunctionTok{rename}\NormalTok{(}\AttributeTok{flowac\_mean =}\NormalTok{ MEAN)}

\NormalTok{calg\_streamdist }\OtherTok{\textless{}{-}} \FunctionTok{read.csv}\NormalTok{(}\StringTok{"Calgary/Processed/zonalstats\_tables/calg\_dist2stream\_min.csv"}\NormalTok{) }\SpecialCharTok{\%\textgreater{}\%} 
\NormalTok{    dplyr}\SpecialCharTok{::}\FunctionTok{select}\NormalTok{(uniqueID, MIN) }\SpecialCharTok{\%\textgreater{}\%} 
  \FunctionTok{rename}\NormalTok{(}\AttributeTok{streamdist\_min =}\NormalTok{ MIN)}

\NormalTok{calg\_dat }\OtherTok{\textless{}{-}}\NormalTok{ calgary\_fishnet }\SpecialCharTok{\%\textgreater{}\%}
  \FunctionTok{mutate}\NormalTok{(}\AttributeTok{uniqueID =} \FunctionTok{as.integer}\NormalTok{(uniqueID)) }\SpecialCharTok{\%\textgreater{}\%} 
  \FunctionTok{left\_join}\NormalTok{(calg\_inund, }\AttributeTok{by =} \StringTok{"uniqueID"}\NormalTok{) }\SpecialCharTok{\%\textgreater{}\%} 
  \FunctionTok{left\_join}\NormalTok{(calg\_pervious, }\AttributeTok{by =} \StringTok{"uniqueID"}\NormalTok{) }\SpecialCharTok{\%\textgreater{}\%}
  \FunctionTok{left\_join}\NormalTok{(calg\_elevation, }\AttributeTok{by =} \StringTok{"uniqueID"}\NormalTok{) }\SpecialCharTok{\%\textgreater{}\%} 
  \FunctionTok{left\_join}\NormalTok{(calg\_flowac, }\AttributeTok{by =} \StringTok{"uniqueID"}\NormalTok{) }\SpecialCharTok{\%\textgreater{}\%} 
  \FunctionTok{left\_join}\NormalTok{(calg\_streamdist, }\AttributeTok{by =} \StringTok{"uniqueID"}\NormalTok{) }\SpecialCharTok{\%\textgreater{}\%} 
  \FunctionTok{mutate}\NormalTok{(}\AttributeTok{flowac\_mean\_log =} \FunctionTok{log}\NormalTok{(flowac\_mean),}
         \AttributeTok{streamdist\_min\_log =} \FunctionTok{log}\NormalTok{(streamdist\_min)) }\SpecialCharTok{\%\textgreater{}\%} 
  \FunctionTok{na.omit}\NormalTok{() }\SpecialCharTok{\%\textgreater{}\%} 
\NormalTok{    dplyr}\SpecialCharTok{::}\FunctionTok{mutate}\NormalTok{(}\AttributeTok{streamdist\_min\_log =} \FunctionTok{if\_else}\NormalTok{(streamdist\_min\_log }\SpecialCharTok{\textless{}}\DecValTok{0}\NormalTok{, }\DecValTok{0}\NormalTok{, streamdist\_min\_log),}
         \AttributeTok{flowac\_mean\_log =} \FunctionTok{if\_else}\NormalTok{(flowac\_mean\_log }\SpecialCharTok{\textless{}}\DecValTok{0}\NormalTok{, }\DecValTok{0}\NormalTok{, flowac\_mean\_log)) }\SpecialCharTok{\%\textgreater{}\%} 
  \FunctionTok{na.omit}\NormalTok{()}
\end{Highlighting}
\end{Shaded}

\begin{Shaded}
\begin{Highlighting}[]
\NormalTok{denv\_pervious }\OtherTok{\textless{}{-}} \FunctionTok{read.csv}\NormalTok{ (}\StringTok{"Denver/Processed/zonalstats\_tables/denv\_pervious\_mean.csv"}\NormalTok{) }\SpecialCharTok{\%\textgreater{}\%} 
\NormalTok{  dplyr}\SpecialCharTok{::}\FunctionTok{select}\NormalTok{(uniqueID, MEAN) }\SpecialCharTok{\%\textgreater{}\%} 
  \FunctionTok{rename}\NormalTok{(}\AttributeTok{pervious\_mean =}\NormalTok{ MEAN)}

\NormalTok{denv\_elevation }\OtherTok{\textless{}{-}} \FunctionTok{read.csv}\NormalTok{(}\StringTok{"Denver/Processed/zonalstats\_tables/denver\_elevation\_mean.csv"}\NormalTok{) }\SpecialCharTok{\%\textgreater{}\%} 
\NormalTok{  dplyr}\SpecialCharTok{::}\FunctionTok{select}\NormalTok{(uniqueID, MEAN) }\SpecialCharTok{\%\textgreater{}\%} 
  \FunctionTok{rename}\NormalTok{(}\AttributeTok{elevation\_mean =}\NormalTok{ MEAN)}


\NormalTok{denv\_flowac }\OtherTok{\textless{}{-}} \FunctionTok{read.csv}\NormalTok{(}\StringTok{"Denver/Processed/zonalstats\_tables/denv\_flowac\_mean.csv"}\NormalTok{) }\SpecialCharTok{\%\textgreater{}\%} 
\NormalTok{    dplyr}\SpecialCharTok{::}\FunctionTok{select}\NormalTok{(uniqueID, MEAN) }\SpecialCharTok{\%\textgreater{}\%} 
  \FunctionTok{rename}\NormalTok{(}\AttributeTok{flowac\_mean =}\NormalTok{ MEAN)}


\NormalTok{denv\_streamdist }\OtherTok{\textless{}{-}} \FunctionTok{read.csv}\NormalTok{(}\StringTok{"Denver/Processed/zonalstats\_tables/denv\_dist2stream\_min.csv"}\NormalTok{) }\SpecialCharTok{\%\textgreater{}\%} 
\NormalTok{    dplyr}\SpecialCharTok{::}\FunctionTok{select}\NormalTok{(uniqueID, MIN) }\SpecialCharTok{\%\textgreater{}\%} 
  \FunctionTok{rename}\NormalTok{(}\AttributeTok{streamdist\_min =}\NormalTok{ MIN)}


\NormalTok{denver\_dat }\OtherTok{\textless{}{-}}\NormalTok{ denver\_fishnet }\SpecialCharTok{\%\textgreater{}\%}
  \FunctionTok{mutate}\NormalTok{(}\AttributeTok{uniqueID =} \FunctionTok{as.integer}\NormalTok{(uniqueID)) }\SpecialCharTok{\%\textgreater{}\%} 
  \FunctionTok{left\_join}\NormalTok{(denv\_pervious, }\AttributeTok{by =} \StringTok{\textquotesingle{}uniqueID\textquotesingle{}}\NormalTok{) }\SpecialCharTok{\%\textgreater{}\%}
  \FunctionTok{left\_join}\NormalTok{(denv\_elevation, }\AttributeTok{by =} \StringTok{\textquotesingle{}uniqueID\textquotesingle{}}\NormalTok{) }\SpecialCharTok{\%\textgreater{}\%} 
  \FunctionTok{left\_join}\NormalTok{(denv\_flowac, }\AttributeTok{by =} \StringTok{\textquotesingle{}uniqueID\textquotesingle{}}\NormalTok{) }\SpecialCharTok{\%\textgreater{}\%} 
  \FunctionTok{left\_join}\NormalTok{(denv\_streamdist, }\AttributeTok{by =} \StringTok{\textquotesingle{}uniqueID\textquotesingle{}}\NormalTok{) }\SpecialCharTok{\%\textgreater{}\%}
  \FunctionTok{mutate}\NormalTok{(}\AttributeTok{flowac\_mean =}\NormalTok{ flowac\_mean}\SpecialCharTok{*}\FloatTok{0.3048}\NormalTok{,}
         \AttributeTok{streamdist\_min =}\NormalTok{ streamdist\_min}\SpecialCharTok{*}\FloatTok{0.3048}\NormalTok{) }\SpecialCharTok{\%\textgreater{}\%} 
    \FunctionTok{mutate}\NormalTok{(}\AttributeTok{flowac\_mean\_log =} \FunctionTok{log}\NormalTok{(flowac\_mean),}
         \AttributeTok{streamdist\_min\_log =} \FunctionTok{log}\NormalTok{(streamdist\_min)) }\SpecialCharTok{\%\textgreater{}\%} 
  \FunctionTok{na.omit}\NormalTok{() }\SpecialCharTok{\%\textgreater{}\%} 
\NormalTok{    dplyr}\SpecialCharTok{::}\FunctionTok{mutate}\NormalTok{(}\AttributeTok{streamdist\_min\_log =} \FunctionTok{if\_else}\NormalTok{(streamdist\_min\_log }\SpecialCharTok{\textless{}}\DecValTok{0}\NormalTok{, }\DecValTok{0}\NormalTok{, streamdist\_min\_log),}
         \AttributeTok{flowac\_mean\_log =} \FunctionTok{if\_else}\NormalTok{(flowac\_mean\_log }\SpecialCharTok{\textless{}}\DecValTok{0}\NormalTok{, }\DecValTok{0}\NormalTok{, flowac\_mean\_log))}
\end{Highlighting}
\end{Shaded}

\begin{block}{Calgary}
\protect\hypertarget{calgary}{}
Below are the features we created for Calgary:

\begin{itemize}
\tightlist
\item
  Flood inundation
\item
  Pervious and impervious landcover
\item
  Elevation
\item
  Flow accumulation
\item
  Distance to river
\end{itemize}

\begin{Shaded}
\begin{Highlighting}[]
\NormalTok{calg\_dat }\OtherTok{\textless{}{-}}\NormalTok{ calg\_dat }\SpecialCharTok{\%\textgreater{}\%}
  \FunctionTok{st\_transform}\NormalTok{(}\AttributeTok{crs =} \DecValTok{3776}\NormalTok{)}

\FunctionTok{ggplot}\NormalTok{() }\SpecialCharTok{+}
  \FunctionTok{geom\_sf}\NormalTok{(}\AttributeTok{data=}\NormalTok{calg\_dat, }\FunctionTok{aes}\NormalTok{(}\AttributeTok{fill=}\FunctionTok{as.factor}\NormalTok{(inund\_sum)), }\AttributeTok{alpha =} \FloatTok{0.8}\NormalTok{, }\AttributeTok{color =} \ConstantTok{NA}\NormalTok{) }\SpecialCharTok{+}
  \FunctionTok{scale\_fill\_manual}\NormalTok{(}\AttributeTok{values =} \FunctionTok{c}\NormalTok{(}\StringTok{"\#CEEBF0"}\NormalTok{, }\StringTok{"\#51A6AE"}\NormalTok{),}
                    \AttributeTok{labels =} \FunctionTok{c}\NormalTok{(}\StringTok{"Not Inundated"}\NormalTok{, }\StringTok{"Inundated"}\NormalTok{),}
                    \AttributeTok{name =} \StringTok{"Observed}\SpecialCharTok{\textbackslash{}n}\StringTok{Flooding"}\NormalTok{) }\SpecialCharTok{+}
  \FunctionTok{labs}\NormalTok{(}\AttributeTok{title=}\StringTok{"Flood Inundation in Calgary"}\NormalTok{,}
       \AttributeTok{subtitle=}\StringTok{"Based on observed flooding in Calgary"}\NormalTok{,}
        \AttributeTok{caption =} \StringTok{"Source: CPLN 675"}\NormalTok{) }\SpecialCharTok{+}
\NormalTok{  mapTheme}
\end{Highlighting}
\end{Shaded}

\includegraphics{luem_a3_markdown_presentation_files/figure-beamer/calgary inundation map-1.pdf}

The Calgary flood inundation raster data we worked with contained four
values: inundated, not inundated, rivers/water bodies, and clouds. We
reclassified this layer so that everything inundated = 1, and everything
else = 0. We then used Zonal Statistics as Table in Arc to calculate the
sum of inundated raster cells for each fishnet cell. Finally, we changed
these sum values into a binary in R. Every fishnet cell that had more
than 16 inundated raster cells became equal to 1, and every other cell
became equal to zero.

**Note: the Calgary inundation raster layer did not cover the entire
city boundary. Therefore, the fishnet cells outside of the raster layer
were removed from further processing.

\begin{Shaded}
\begin{Highlighting}[]
\FunctionTok{ggplot}\NormalTok{() }\SpecialCharTok{+}
  \FunctionTok{geom\_sf}\NormalTok{(}\AttributeTok{data=}\NormalTok{calg\_dat, }\FunctionTok{aes}\NormalTok{(}\AttributeTok{fill=}\FunctionTok{factor}\NormalTok{(}\FunctionTok{ntile}\NormalTok{(streamdist\_min, }\DecValTok{4}\NormalTok{))), }
            \AttributeTok{colour=}\ConstantTok{NA}\NormalTok{) }\SpecialCharTok{+}
  \FunctionTok{scale\_fill\_manual}\NormalTok{(}\AttributeTok{values =}\NormalTok{ yellows,}
                    \AttributeTok{labels=} \FunctionTok{as.character}\NormalTok{(}\FunctionTok{round}\NormalTok{(}\FunctionTok{quantile}\NormalTok{(calg\_dat}\SpecialCharTok{$}\NormalTok{streamdist\_min,}
                                                 \FunctionTok{c}\NormalTok{(}\FloatTok{0.2}\NormalTok{,.}\DecValTok{4}\NormalTok{,.}\DecValTok{6}\NormalTok{,.}\DecValTok{8}\NormalTok{),}
                                                 \AttributeTok{na.rm=}\NormalTok{T))),}
                    \AttributeTok{name =} \StringTok{"Distance}\SpecialCharTok{\textbackslash{}n}\StringTok{(Quantile Breaks}\SpecialCharTok{\textbackslash{}n}\StringTok{in Meters)"}\NormalTok{) }\SpecialCharTok{+}
  \FunctionTok{labs}\NormalTok{(}\AttributeTok{title=}\StringTok{"Distance from Rivers in Calgary"}\NormalTok{,}
      \AttributeTok{subtitle=}\StringTok{"Based on Calgary Hydrology Data"}\NormalTok{,}
      \AttributeTok{caption =} \StringTok{"Source: data.calgary.ca"}\NormalTok{) }\SpecialCharTok{+}
\NormalTok{  mapTheme                    }
\end{Highlighting}
\end{Shaded}

\includegraphics{luem_a3_markdown_presentation_files/figure-beamer/calgary stream dist map-1.pdf}

We used the distance accumulation tool to create a raster layered where
each pixel was equal to its distance from a river. We then set each
fishnet cell in the resulting table equal to the minimum distance to a
river in that cell. We thought it best to over-sample cells near rivers
rather than under-sample, because our model is based on riverine
flooding.

\begin{Shaded}
\begin{Highlighting}[]
\FunctionTok{ggplot}\NormalTok{() }\SpecialCharTok{+}
  \FunctionTok{geom\_sf}\NormalTok{(}\AttributeTok{data=}\NormalTok{calg\_dat, }\FunctionTok{aes}\NormalTok{(}\AttributeTok{fill=}\FunctionTok{factor}\NormalTok{(}\FunctionTok{ntile}\NormalTok{(flowac\_mean,}\DecValTok{4}\NormalTok{))), }
            \AttributeTok{colour=}\ConstantTok{NA}\NormalTok{) }\SpecialCharTok{+}
  \FunctionTok{scale\_fill\_manual}\NormalTok{(}\AttributeTok{values =}\NormalTok{ blues,}
                    \AttributeTok{labels=} \FunctionTok{as.character}\NormalTok{(}\FunctionTok{round}\NormalTok{(}\FunctionTok{quantile}\NormalTok{(calg\_dat}\SpecialCharTok{$}\NormalTok{flowac\_mean,}
                                                 \FunctionTok{c}\NormalTok{(}\FloatTok{0.2}\NormalTok{,.}\DecValTok{4}\NormalTok{,.}\DecValTok{6}\NormalTok{,.}\DecValTok{8}\NormalTok{),}
                                                 \AttributeTok{na.rm=}\NormalTok{T))),}
                    \AttributeTok{name =} \StringTok{"Mean Flow}\SpecialCharTok{\textbackslash{}n}\StringTok{Accumulation}\SpecialCharTok{\textbackslash{}n}\StringTok{(Quantile Breaks)"}\NormalTok{) }\SpecialCharTok{+}
  \FunctionTok{labs}\NormalTok{(}\AttributeTok{title=}\StringTok{"Precipitation Flow Accumulation in Calgary"}\NormalTok{,}
       \AttributeTok{subtitle=}\StringTok{"Based on Calgary Elevation Data"}\NormalTok{,}
        \AttributeTok{caption =} \StringTok{"Source: 18M DEM, CPLN 675"}\NormalTok{) }\SpecialCharTok{+}
\NormalTok{  mapTheme}
\end{Highlighting}
\end{Shaded}

\includegraphics{luem_a3_markdown_presentation_files/figure-beamer/calgary fac map-1.pdf}

We used the DEM elevation layer to calculate flow accumulation in
ArcGIS. Flow accumulation approximates where water that falls within
city boundaries during a rain event would flow to by assigning each
raster cell a value equal to the amount of cells that ``flow'' into it
(based on elevation and flow direction). We then set each fishnet cell
equal to the mean flow accumulation value it contained.

\begin{Shaded}
\begin{Highlighting}[]
\FunctionTok{ggplot}\NormalTok{() }\SpecialCharTok{+}
  \FunctionTok{geom\_sf}\NormalTok{(}\AttributeTok{data=}\NormalTok{calg\_dat, }\FunctionTok{aes}\NormalTok{(}\AttributeTok{fill=}\FunctionTok{factor}\NormalTok{(}\FunctionTok{ntile}\NormalTok{(pervious\_mean,}\DecValTok{4}\NormalTok{))), }
            \AttributeTok{colour=}\ConstantTok{NA}\NormalTok{) }\SpecialCharTok{+}
  \FunctionTok{scale\_fill\_manual}\NormalTok{(}\AttributeTok{values =}\NormalTok{ greens,}
                    \AttributeTok{labels=} \FunctionTok{as.character}\NormalTok{(}\FunctionTok{round}\NormalTok{(}\FunctionTok{quantile}\NormalTok{(calg\_dat}\SpecialCharTok{$}\NormalTok{pervious\_mean,}
                                                 \FunctionTok{c}\NormalTok{(}\FloatTok{0.2}\NormalTok{,.}\DecValTok{4}\NormalTok{,.}\DecValTok{6}\NormalTok{,.}\DecValTok{8}\NormalTok{),}
                                                 \AttributeTok{na.rm=}\NormalTok{T), }\DecValTok{2}\NormalTok{)),}
                    \AttributeTok{name =} \StringTok{"Pervious}\SpecialCharTok{\textbackslash{}n}\StringTok{Surface}\SpecialCharTok{\textbackslash{}n}\StringTok{(Quantile}\SpecialCharTok{\textbackslash{}n}\StringTok{Breaks)"}\NormalTok{) }\SpecialCharTok{+}
  \FunctionTok{labs}\NormalTok{(}\AttributeTok{title=}\StringTok{"Pervious Surface in Calgary"}\NormalTok{,}
       \AttributeTok{subtitle=}\StringTok{"Based on Calgary Land Cover Data"}\NormalTok{,}
        \AttributeTok{caption =} \StringTok{"Source: data.calgary.ca"}\NormalTok{) }\SpecialCharTok{+}
\NormalTok{  mapTheme}
\end{Highlighting}
\end{Shaded}

\includegraphics{luem_a3_markdown_presentation_files/figure-beamer/calgary pervious surface map-1.pdf}

We reclassified land cover raster data in Arc into two categories
pervious surface = 1, and impervious surface = 0. We then set each
fishnet cell in the resulting table equal to the mean of the pervious
raster cells it contained. This was to account for a high amount of
local variation in permeability.

\begin{Shaded}
\begin{Highlighting}[]
\FunctionTok{ggplot}\NormalTok{() }\SpecialCharTok{+}
  \FunctionTok{geom\_sf}\NormalTok{(}\AttributeTok{data=}\NormalTok{calg\_dat, }\FunctionTok{aes}\NormalTok{(}\AttributeTok{fill=}\FunctionTok{factor}\NormalTok{(}\FunctionTok{ntile}\NormalTok{(elevation\_mean,}\DecValTok{4}\NormalTok{))), }
            \AttributeTok{colour=}\ConstantTok{NA}\NormalTok{) }\SpecialCharTok{+}
  \FunctionTok{scale\_fill\_manual}\NormalTok{(}\AttributeTok{values =}\NormalTok{ neutrals,}
                    \AttributeTok{labels=} \FunctionTok{as.character}\NormalTok{(}\FunctionTok{round}\NormalTok{(}\FunctionTok{quantile}\NormalTok{(calg\_dat}\SpecialCharTok{$}\NormalTok{elevation\_mean,}
                                                         \FunctionTok{c}\NormalTok{(}\FloatTok{0.2}\NormalTok{,.}\DecValTok{4}\NormalTok{,.}\DecValTok{6}\NormalTok{,.}\DecValTok{8}\NormalTok{),}
                                                         \AttributeTok{na.rm=}\NormalTok{T), }\DecValTok{2}\NormalTok{)),}
                    \AttributeTok{name =} \StringTok{"Mean Elevation}\SpecialCharTok{\textbackslash{}n}\StringTok{(Quantile Breaks}\SpecialCharTok{\textbackslash{}n}\StringTok{Categorical)"}\NormalTok{) }\SpecialCharTok{+}
  \FunctionTok{labs}\NormalTok{(}\AttributeTok{title=}\StringTok{"Elevation in Calgary"}\NormalTok{,}
        \AttributeTok{caption =} \StringTok{"Source: CPLN 675"}\NormalTok{) }\SpecialCharTok{+}
\NormalTok{    mapTheme}
\end{Highlighting}
\end{Shaded}

\includegraphics{luem_a3_markdown_presentation_files/figure-beamer/calgary elevation map-1.pdf}

For elevation, we simply used the DEM for Calgary and set each fishnet
cell equal to the median elevation in that cell.
\end{block}

\begin{block}{Denver}
\protect\hypertarget{denver}{}
We performed the exact same operations for the Denver features,
excluding inundation which we're setting out to predict.

\begin{Shaded}
\begin{Highlighting}[]
\NormalTok{denver\_dat }\OtherTok{\textless{}{-}}\NormalTok{ denver\_dat }\SpecialCharTok{\%\textgreater{}\%}
  \FunctionTok{st\_transform}\NormalTok{(}\AttributeTok{crs =} \DecValTok{2232}\NormalTok{)}
\end{Highlighting}
\end{Shaded}

\begin{Shaded}
\begin{Highlighting}[]
\FunctionTok{ggplot}\NormalTok{() }\SpecialCharTok{+}
  \FunctionTok{geom\_sf}\NormalTok{(}\AttributeTok{data=}\NormalTok{denver\_dat, }\FunctionTok{aes}\NormalTok{(}\AttributeTok{fill=}\FunctionTok{factor}\NormalTok{(}\FunctionTok{ntile}\NormalTok{(streamdist\_min,}\DecValTok{4}\NormalTok{))), }
            \AttributeTok{colour=}\ConstantTok{NA}\NormalTok{) }\SpecialCharTok{+}
  \FunctionTok{scale\_fill\_manual}\NormalTok{(}\AttributeTok{values =}\NormalTok{ yellows,}
                    \AttributeTok{labels=} \FunctionTok{as.character}\NormalTok{(}\FunctionTok{round}\NormalTok{(}\FunctionTok{quantile}\NormalTok{(denver\_dat}\SpecialCharTok{$}\NormalTok{streamdist\_min,                                                    }\FunctionTok{c}\NormalTok{(}\FloatTok{0.2}\NormalTok{,.}\DecValTok{4}\NormalTok{,.}\DecValTok{6}\NormalTok{,.}\DecValTok{8}\NormalTok{),}
                                        \AttributeTok{na.rm=}\NormalTok{T), }\DecValTok{2}\NormalTok{)),}
     \AttributeTok{name =} \StringTok{"Distance}\SpecialCharTok{\textbackslash{}n}\StringTok{(Quantile Breaks}\SpecialCharTok{\textbackslash{}n}\StringTok{ in Metres)"}\NormalTok{) }\SpecialCharTok{+}
  \FunctionTok{labs}\NormalTok{(}\AttributeTok{title=}\StringTok{"Distance from Rivers in Denver"}\NormalTok{,}
       \AttributeTok{subtitle=}\StringTok{"Based on Denver Hydrology Data"}\NormalTok{,}
        \AttributeTok{caption =} \StringTok{"Source: Denver Open Data Catalog"}\NormalTok{) }\SpecialCharTok{+}
\NormalTok{  mapTheme}
\end{Highlighting}
\end{Shaded}

\includegraphics{luem_a3_markdown_presentation_files/figure-beamer/denver stream dist map-1.pdf}

\begin{Shaded}
\begin{Highlighting}[]
\FunctionTok{ggplot}\NormalTok{() }\SpecialCharTok{+}
  \FunctionTok{geom\_sf}\NormalTok{(}\AttributeTok{data=}\NormalTok{denver\_dat, }\FunctionTok{aes}\NormalTok{(}\AttributeTok{fill=}\FunctionTok{factor}\NormalTok{(}\FunctionTok{ntile}\NormalTok{(flowac\_mean,}\DecValTok{4}\NormalTok{))), }
            \AttributeTok{colour=}\ConstantTok{NA}\NormalTok{) }\SpecialCharTok{+}
  \FunctionTok{scale\_fill\_manual}\NormalTok{(}\AttributeTok{values =}\NormalTok{ blues,}
                    \AttributeTok{labels=} \FunctionTok{as.character}\NormalTok{(}\FunctionTok{round}\NormalTok{(}\FunctionTok{quantile}\NormalTok{(denver\_dat}\SpecialCharTok{$}\NormalTok{flowac\_mean,}
                                                 \FunctionTok{c}\NormalTok{(}\FloatTok{0.2}\NormalTok{,.}\DecValTok{4}\NormalTok{,.}\DecValTok{6}\NormalTok{,.}\DecValTok{8}\NormalTok{),}
                                                 \AttributeTok{na.rm=}\NormalTok{T))),}
                    \AttributeTok{name =} \StringTok{"Mean Flow}\SpecialCharTok{\textbackslash{}n}\StringTok{Accumulation}\SpecialCharTok{\textbackslash{}n}\StringTok{(Quantile Breaks)"}\NormalTok{) }\SpecialCharTok{+}
  \FunctionTok{labs}\NormalTok{(}\AttributeTok{title=}\StringTok{"Precipitation Flow Accumulation in Denver"}\NormalTok{,}
       \AttributeTok{subtitle=}\StringTok{"Based on Denver Elevation Data"}\NormalTok{,}
        \AttributeTok{caption =} \StringTok{"Source: webgis.com"}\NormalTok{) }\SpecialCharTok{+}
\NormalTok{  mapTheme}
\end{Highlighting}
\end{Shaded}

\includegraphics{luem_a3_markdown_presentation_files/figure-beamer/denver fac map-1.pdf}

\begin{Shaded}
\begin{Highlighting}[]
\FunctionTok{ggplot}\NormalTok{() }\SpecialCharTok{+}
  \FunctionTok{geom\_sf}\NormalTok{(}\AttributeTok{data=}\NormalTok{denver\_dat, }\FunctionTok{aes}\NormalTok{(}\AttributeTok{fill=}\FunctionTok{factor}\NormalTok{(}\FunctionTok{ntile}\NormalTok{(pervious\_mean,}\DecValTok{4}\NormalTok{))), }
            \AttributeTok{colour=}\ConstantTok{NA}\NormalTok{) }\SpecialCharTok{+}
  \FunctionTok{scale\_fill\_manual}\NormalTok{(}\AttributeTok{values =}\NormalTok{ greens,}
                    \AttributeTok{labels=} \FunctionTok{as.character}\NormalTok{(}\FunctionTok{round}\NormalTok{(}\FunctionTok{quantile}\NormalTok{(denver\_dat}\SpecialCharTok{$}\NormalTok{pervious\_mean,}
                                                 \FunctionTok{c}\NormalTok{(}\FloatTok{0.2}\NormalTok{,.}\DecValTok{4}\NormalTok{,.}\DecValTok{6}\NormalTok{,.}\DecValTok{8}\NormalTok{),}
                                                 \AttributeTok{na.rm=}\NormalTok{T), }\DecValTok{2}\NormalTok{)),}
                    \AttributeTok{name =} \StringTok{"Permeable}\SpecialCharTok{\textbackslash{}n}\StringTok{Surface}\SpecialCharTok{\textbackslash{}n}\StringTok{(Quantile}\SpecialCharTok{\textbackslash{}n}\StringTok{Breaks)"}\NormalTok{) }\SpecialCharTok{+}
  \FunctionTok{labs}\NormalTok{(}\AttributeTok{title=}\StringTok{"Permeable Surface in Denver"}\NormalTok{,}
       \AttributeTok{subtitle=}\StringTok{"Based on Denver Land Cover Data"}\NormalTok{,}
        \AttributeTok{caption =} \StringTok{"Source: Denver Regional Council of Governments Regional Data Catalog"}\NormalTok{) }\SpecialCharTok{+}
\NormalTok{  mapTheme}
\end{Highlighting}
\end{Shaded}

\includegraphics{luem_a3_markdown_presentation_files/figure-beamer/denver pervious surface map-1.pdf}

\begin{Shaded}
\begin{Highlighting}[]
\FunctionTok{ggplot}\NormalTok{() }\SpecialCharTok{+}
  \FunctionTok{geom\_sf}\NormalTok{(}\AttributeTok{data=}\NormalTok{denver\_dat, }\FunctionTok{aes}\NormalTok{(}\AttributeTok{fill=}\FunctionTok{factor}\NormalTok{(}\FunctionTok{ntile}\NormalTok{(elevation\_mean,}\DecValTok{4}\NormalTok{))), }
            \AttributeTok{colour=}\ConstantTok{NA}\NormalTok{) }\SpecialCharTok{+}
  \FunctionTok{scale\_fill\_manual}\NormalTok{(}\AttributeTok{values =}\NormalTok{ neutrals,}
                    \AttributeTok{labels=} \FunctionTok{as.character}\NormalTok{(}\FunctionTok{round}\NormalTok{(}\FunctionTok{quantile}\NormalTok{(denver\_dat}\SpecialCharTok{$}\NormalTok{elevation\_mean,}
                                                 \FunctionTok{c}\NormalTok{(}\FloatTok{0.2}\NormalTok{,.}\DecValTok{4}\NormalTok{,.}\DecValTok{6}\NormalTok{,.}\DecValTok{8}\NormalTok{),}
                                                 \AttributeTok{na.rm=}\NormalTok{T))),}
                    \AttributeTok{name =} \StringTok{"Mean Elevation}\SpecialCharTok{\textbackslash{}n}\StringTok{(Quantile Breaks}\SpecialCharTok{\textbackslash{}n}\StringTok{Categorical)"}\NormalTok{) }\SpecialCharTok{+}
  \FunctionTok{labs}\NormalTok{(}\AttributeTok{title=}\StringTok{"Elevation in Denver"}\NormalTok{,}
        \AttributeTok{caption =} \StringTok{"Source: webgis.com"}\NormalTok{) }\SpecialCharTok{+}
\NormalTok{  mapTheme}
\end{Highlighting}
\end{Shaded}

\includegraphics{luem_a3_markdown_presentation_files/figure-beamer/denver elevation map-1.pdf}
\end{block}
\end{block}

\begin{block}{2.4 What is our Data Telling Us?}
\protect\hypertarget{what-is-our-data-telling-us}{}
How do our variables correlate with inundated/not inundated values?

Note: we had to log adjust our Flow Accumulation and Distance to Stream
variables to compensate the wide range of outliers in the data set.

\begin{Shaded}
\begin{Highlighting}[]
\NormalTok{calg\_PlotVariables }\OtherTok{\textless{}{-}}\NormalTok{ calg\_dat }\SpecialCharTok{\%\textgreater{}\%} 
  \FunctionTok{as.data.frame}\NormalTok{() }\SpecialCharTok{\%\textgreater{}\%}
\NormalTok{    dplyr}\SpecialCharTok{::}\FunctionTok{select}\NormalTok{(inund\_sum, pervious\_mean, elevation\_mean, flowac\_mean\_log, streamdist\_min\_log) }\SpecialCharTok{\%\textgreater{}\%} 
    \FunctionTok{pivot\_longer}\NormalTok{(}\AttributeTok{cols =} \SpecialCharTok{{-}}\NormalTok{inund\_sum)}
\end{Highlighting}
\end{Shaded}

The violin plots below shows how each variable is spread across 0/1
values of inundation. They visually indicate the relationship between
the variables we assumed would cause / correlate to flooding, as a
function of how we processed them with relation to Calgary's fishnet.
For instance:

\begin{itemize}
\item
  Fishnet cells at lower elevations are more inundated than those at
  higher elevations.
\item
  Fishnet cells at longer distances from a stream are not as inundated
  as those that are at shorter distances from a stream.
\end{itemize}

By confirming the hypotheses we started with, this quick plot gives us
confidence in our variable selection before we proceed with building
regression models.

\begin{Shaded}
\begin{Highlighting}[]
\CommentTok{\#violin plots}
\CommentTok{\#change code for for calgary\_dat}
\DocumentationTok{\#\#use boxplots with scatter points to visualize the spread of data?}

\FunctionTok{ggplot}\NormalTok{(calg\_PlotVariables) }\SpecialCharTok{+} 
     \FunctionTok{geom\_violin}\NormalTok{(}\FunctionTok{aes}\NormalTok{(}\AttributeTok{x =} \FunctionTok{as.factor}\NormalTok{(inund\_sum), }
                  \AttributeTok{y =}\NormalTok{ value, }\AttributeTok{fill =} \FunctionTok{as.factor}\NormalTok{(inund\_sum))) }\SpecialCharTok{+} 
     \FunctionTok{facet\_wrap}\NormalTok{(}\SpecialCharTok{\textasciitilde{}}\NormalTok{name, }\AttributeTok{scales =} \StringTok{"free\_y"}\NormalTok{) }\SpecialCharTok{+}
     \FunctionTok{labs}\NormalTok{(}\AttributeTok{x=}\StringTok{"Inundated"}\NormalTok{, }\AttributeTok{y=}\StringTok{"Value"}\NormalTok{) }\SpecialCharTok{+} 
     \FunctionTok{scale\_fill\_manual}\NormalTok{(}\AttributeTok{values =} \FunctionTok{c}\NormalTok{(}\StringTok{"\#CEEBF0"}\NormalTok{, }\StringTok{"\#51A6AE"}\NormalTok{),}
     \AttributeTok{labels =} \FunctionTok{c}\NormalTok{(}\StringTok{"Not Inundated"}\NormalTok{,}\StringTok{"Inundated"}\NormalTok{), }\AttributeTok{name =} \StringTok{""}\NormalTok{) }\SpecialCharTok{+}
     \FunctionTok{labs}\NormalTok{(}\AttributeTok{x=}\StringTok{"Inundated"}\NormalTok{, }\AttributeTok{y=}\StringTok{"Value"}\NormalTok{) }\SpecialCharTok{+} 
\NormalTok{  plotTheme}
\end{Highlighting}
\end{Shaded}

\includegraphics{luem_a3_markdown_presentation_files/figure-beamer/data spread-1.pdf}

\begin{Shaded}
\begin{Highlighting}[]
\DocumentationTok{\#\#facet\_wrap {-} one ggplot recipe for each variable}
\DocumentationTok{\#\#\#use scales = free or free\_y to plot values that are comparatively lower or higher}
\end{Highlighting}
\end{Shaded}

\textbf{How many fishnet cells in Calgary are inundated?}

A quick calculation reveals that 509 of 5373, or 9.5\% of fishnet cells
in Calgary are inundated.

\begin{Shaded}
\begin{Highlighting}[]
\NormalTok{calg\_inund\_fishnet }\OtherTok{\textless{}{-}}\NormalTok{ calg\_dat }\SpecialCharTok{\%\textgreater{}\%} 
  \FunctionTok{filter}\NormalTok{(inund\_sum }\SpecialCharTok{==} \DecValTok{1}\NormalTok{)}

\NormalTok{no\_fishnets }\OtherTok{\textless{}{-}}\NormalTok{(}\DecValTok{509}\SpecialCharTok{/}\DecValTok{5373}\NormalTok{)}\SpecialCharTok{*}\DecValTok{100}
\end{Highlighting}
\end{Shaded}
\end{block}
\end{frame}

\begin{frame}[fragile]{3. Logistic Regressions}
\protect\hypertarget{logistic-regressions}{}
\begin{block}{3.1 Test, Train, Correlate, Regress: Model 1}
\protect\hypertarget{test-train-correlate-regress-model-1}{}
Our first binomial regression model inputs the Calgary training set with
the dependent variable set to ``Inundation'', and four independent
variables, namely, Land Porosity (a binary derived from land cover),
Elevation, Flow Accumulation, and Distance to a Stream.

\begin{Shaded}
\begin{Highlighting}[]
\FunctionTok{set.seed}\NormalTok{(}\DecValTok{3456}\NormalTok{)}

\NormalTok{trainIndex }\OtherTok{\textless{}{-}} \FunctionTok{createDataPartition}\NormalTok{(calg\_dat}\SpecialCharTok{$}\NormalTok{elevation\_mean, }\AttributeTok{p =}\NormalTok{ .}\DecValTok{70}\NormalTok{,}
                                  \AttributeTok{list =} \ConstantTok{FALSE}\NormalTok{,}
                                  \AttributeTok{times =} \DecValTok{1}\NormalTok{) }

\NormalTok{inundTrain }\OtherTok{\textless{}{-}}\NormalTok{ calg\_dat[ trainIndex,] }\SpecialCharTok{\%\textgreater{}\%} 
\NormalTok{  dplyr}\SpecialCharTok{::}\FunctionTok{select}\NormalTok{(}\SpecialCharTok{{-}}\NormalTok{flowac\_mean, }\SpecialCharTok{{-}}\NormalTok{streamdist\_min)}

\NormalTok{inundTest  }\OtherTok{\textless{}{-}}\NormalTok{ calg\_dat[}\SpecialCharTok{{-}}\NormalTok{trainIndex,]}\SpecialCharTok{\%\textgreater{}\%} 
\NormalTok{  dplyr}\SpecialCharTok{::}\FunctionTok{select}\NormalTok{(}\SpecialCharTok{{-}}\NormalTok{flowac\_mean, }\SpecialCharTok{{-}}\NormalTok{streamdist\_min)}

\DocumentationTok{\#\#the sets are randomly generated}
\DocumentationTok{\#\#p=0.70 indicates the 70/30 partition}
\end{Highlighting}
\end{Shaded}

A quick correlation test reveals that there isn't too high a level of
multi-collinearity between the variables we have selected for our
regression.

\begin{Shaded}
\begin{Highlighting}[]
\NormalTok{corr }\OtherTok{\textless{}{-}}\NormalTok{ calg\_dat }\SpecialCharTok{\%\textgreater{}\%}
  \FunctionTok{as.data.frame}\NormalTok{() }\SpecialCharTok{\%\textgreater{}\%}
\NormalTok{  dplyr}\SpecialCharTok{::}\FunctionTok{select}\NormalTok{(inund\_sum, pervious\_mean, elevation\_mean, flowac\_mean\_log, streamdist\_min\_log) }\SpecialCharTok{\%\textgreater{}\%} 
  \FunctionTok{rename}\NormalTok{(}\StringTok{"Distance to Stream (log)"} \OtherTok{=}\NormalTok{ streamdist\_min\_log,}
         \StringTok{"Flow Accumulation (log)"} \OtherTok{=}\NormalTok{ flowac\_mean\_log,}
         \StringTok{"Elevation"} \OtherTok{=}\NormalTok{ elevation\_mean,}
         \StringTok{"Land Porosity"} \OtherTok{=}\NormalTok{ pervious\_mean,}
         \StringTok{"Inundation (observed)"} \OtherTok{=}\NormalTok{ inund\_sum)}


\NormalTok{calg\_matrix }\OtherTok{=} \FunctionTok{cor}\NormalTok{(corr)}

\FunctionTok{ggcorrplot}\NormalTok{(calg\_matrix, }\AttributeTok{method=}\StringTok{"square"}\NormalTok{, }\AttributeTok{colors =} \FunctionTok{c}\NormalTok{(}\StringTok{"\#73BBBF"}\NormalTok{, }\StringTok{"\#FDF4E9"}\NormalTok{, }\StringTok{"\#92AF7E"}\NormalTok{),}
           \AttributeTok{tl.cex=}\DecValTok{7}\NormalTok{)}
\end{Highlighting}
\end{Shaded}

\includegraphics{luem_a3_markdown_presentation_files/figure-beamer/correlation matrix-1.pdf}

A logistic regression on the first model displays the following results:

\begin{itemize}
\item
  The p-values indicate a high level of statistical significance for
  three out of four of our independent variables -- elevation, flow
  accumulation, and stream distance.
\item
  The Aikake Information Criterion (AIC) is a useful number when
  comparing models on the basis of their performance. When selecting
  multiple regression models, a lower AIC indicates a better model
  performance.\textsuperscript{3} In this case, the AIC is
  \textbf{1259.8}.
\item
  In our next step, we exponentiate the coefficients to interpret their
  relationship to the likelihood of flood inundation.
\end{itemize}

\begin{Shaded}
\begin{Highlighting}[]
\NormalTok{inundModel }\OtherTok{\textless{}{-}} \FunctionTok{glm}\NormalTok{(inund\_sum }\SpecialCharTok{\textasciitilde{}}\NormalTok{ ., }
                    \AttributeTok{family=}\StringTok{"binomial"}\NormalTok{(}\AttributeTok{link=}\StringTok{"logit"}\NormalTok{), }\AttributeTok{data =}\NormalTok{ inundTrain }\SpecialCharTok{\%\textgreater{}\%}
                                                            \FunctionTok{as.data.frame}\NormalTok{() }\SpecialCharTok{\%\textgreater{}\%}
\NormalTok{                                                            dplyr}\SpecialCharTok{::}\FunctionTok{select}\NormalTok{(}\SpecialCharTok{{-}}\NormalTok{geometry, }\SpecialCharTok{{-}}\NormalTok{uniqueID))}
\FunctionTok{summary}\NormalTok{(inundModel)}
\end{Highlighting}
\end{Shaded}

\begin{verbatim}
## 
## Call:
## glm(formula = inund_sum ~ ., family = binomial(link = "logit"), 
##     data = inundTrain %>% as.data.frame() %>% dplyr::select(-geometry, 
##         -uniqueID))
## 
## Deviance Residuals: 
##      Min        1Q    Median        3Q       Max  
## -1.97219  -0.26735  -0.15085  -0.06525   2.96692  
## 
## Coefficients:
##                    Estimate Std. Error z value             Pr(>|z|)    
## (Intercept)         1.81143    0.40791   4.441           0.00000896 ***
## pervious_mean      -0.36485    0.30667  -1.190              0.23417    
## elevation_mean     -0.63572    0.06873  -9.250 < 0.0000000000000002 ***
## flowac_mean_log     0.10099    0.03350   3.014              0.00258 ** 
## streamdist_min_log -0.52888    0.02738 -19.319 < 0.0000000000000002 ***
## ---
## Signif. codes:  0 '***' 0.001 '**' 0.01 '*' 0.05 '.' 0.1 ' ' 1
## 
## (Dispersion parameter for binomial family taken to be 1)
## 
##     Null deviance: 2296.9  on 3762  degrees of freedom
## Residual deviance: 1249.8  on 3758  degrees of freedom
## AIC: 1259.8
## 
## Number of Fisher Scoring iterations: 7
\end{verbatim}

The exponentiated coefficients of the first model exhibit the following
relationships between the dependent and independent variables:

\begin{itemize}
\tightlist
\item
  All else equal, a unit increase in Land Porosity \emph{reduces} the
  chances of inundation by 30.56\% (this is awkward because Land
  Porosity is also a binary variable)
\item
  All else equal, a unit increase in Elevation \emph{reduces} the
  chances of flood inundation by 47\%
\item
  All else equal, a unit change in Flow Accumulation \emph{increases}
  the odds of flood inundation by 13.9\%
\item
  All else equal, a unit increase in Distance to Stream \emph{reduces}
  the odds of flood inundation by 41.1\%
\end{itemize}

\begin{Shaded}
\begin{Highlighting}[]
\DocumentationTok{\#\# \% change in Y for unit change in X = [(exponent (coefficient of X) {-} 1)] * 100}
\DocumentationTok{\#\# {-} or + sign indicates associated increase or decrease}

\NormalTok{inundModel}\SpecialCharTok{$}\NormalTok{coefficients}

\NormalTok{inundModel\_vars }\OtherTok{\textless{}{-}} \FunctionTok{c}\NormalTok{(}\StringTok{"Land Porosity"}\NormalTok{, }\StringTok{"Elevation"}\NormalTok{, }\StringTok{"Flow Accumulation (log)"}\NormalTok{, }\StringTok{"Distance to Stream (log)"}\NormalTok{)}
\NormalTok{inundModel\_coeffs }\OtherTok{\textless{}{-}} \FunctionTok{c}\NormalTok{(((}\FunctionTok{exp}\NormalTok{(}\SpecialCharTok{{-}}\FloatTok{0.3648453}\NormalTok{ ) }\SpecialCharTok{{-}} \DecValTok{1}\NormalTok{) }\SpecialCharTok{*} \DecValTok{100}\NormalTok{), ((}\FunctionTok{exp}\NormalTok{(}\SpecialCharTok{{-}}\FloatTok{0.6357177}\NormalTok{) }\SpecialCharTok{{-}} \DecValTok{1}\NormalTok{) }\SpecialCharTok{*} \DecValTok{100}\NormalTok{), ((}\FunctionTok{exp}\NormalTok{(}\FloatTok{0.1009900}\NormalTok{) }\SpecialCharTok{{-}} \DecValTok{1}\NormalTok{) }\SpecialCharTok{*} \DecValTok{100}\NormalTok{), ((}\FunctionTok{exp}\NormalTok{( }\SpecialCharTok{{-}}\FloatTok{0.5288784}\NormalTok{) }\SpecialCharTok{{-}} \DecValTok{1}\NormalTok{) }\SpecialCharTok{*} \DecValTok{100}\NormalTok{))}

\NormalTok{inundModel\_coefficients }\OtherTok{\textless{}{-}} \FunctionTok{data.frame}\NormalTok{(inundModel\_vars, inundModel\_coeffs)}

\NormalTok{inundModel\_coefficients }\SpecialCharTok{\%\textgreater{}\%} 
  \FunctionTok{kbl}\NormalTok{(}\AttributeTok{caption =} \StringTok{"Exponentiated Coefficients: Logistic Regression Model 1"}\NormalTok{) }\SpecialCharTok{\%\textgreater{}\%} 
   \FunctionTok{kable\_styling}\NormalTok{(}\AttributeTok{bootstrap\_options =} \StringTok{"striped"}\NormalTok{, }\AttributeTok{full\_width =}\NormalTok{ F, }\AttributeTok{position =} \StringTok{"left"}\NormalTok{)}
\end{Highlighting}
\end{Shaded}
\end{block}

\begin{block}{3.2 Regress: Model 2 \& 3}
\protect\hypertarget{regress-model-2-3}{}
Let's test out the results of some other models:

\textbf{Model 2: Does `Land Porosity' matter?}

We eliminate Land Porosity, which did not exhibit statistical
significance in our previous model.

\begin{itemize}
\item
  The resulting model shows that all independent variables are
  statistically significant.
\item
  The AIC is very marginally lower than our first model, at
  \textbf{1259.2}.
\end{itemize}

\begin{Shaded}
\begin{Highlighting}[]
\NormalTok{inundTrain\_2 }\OtherTok{\textless{}{-}}\NormalTok{ calg\_dat[ trainIndex,] }\SpecialCharTok{\%\textgreater{}\%} 
\NormalTok{  dplyr}\SpecialCharTok{::}\FunctionTok{select}\NormalTok{(}\SpecialCharTok{{-}}\NormalTok{flowac\_mean, }\SpecialCharTok{{-}}\NormalTok{streamdist\_min, }\SpecialCharTok{{-}}\NormalTok{pervious\_mean)}

\NormalTok{inundTest\_2  }\OtherTok{\textless{}{-}}\NormalTok{ calg\_dat[}\SpecialCharTok{{-}}\NormalTok{trainIndex,]}\SpecialCharTok{\%\textgreater{}\%} 
\NormalTok{  dplyr}\SpecialCharTok{::}\FunctionTok{select}\NormalTok{(}\SpecialCharTok{{-}}\NormalTok{flowac\_mean, }\SpecialCharTok{{-}}\NormalTok{streamdist\_min, }\SpecialCharTok{{-}}\NormalTok{pervious\_mean)}


\NormalTok{inundModel\_2 }\OtherTok{\textless{}{-}} \FunctionTok{glm}\NormalTok{(inund\_sum }\SpecialCharTok{\textasciitilde{}}\NormalTok{ ., }
                    \AttributeTok{family=}\StringTok{"binomial"}\NormalTok{(}\AttributeTok{link=}\StringTok{"logit"}\NormalTok{), }\AttributeTok{data =}\NormalTok{ inundTrain\_2 }\SpecialCharTok{\%\textgreater{}\%}
                                                            \FunctionTok{as.data.frame}\NormalTok{() }\SpecialCharTok{\%\textgreater{}\%}
\NormalTok{                                                            dplyr}\SpecialCharTok{::}\FunctionTok{select}\NormalTok{(}\SpecialCharTok{{-}}\NormalTok{geometry, }\SpecialCharTok{{-}}\NormalTok{uniqueID))}
\FunctionTok{summary}\NormalTok{(inundModel\_2)}
\end{Highlighting}
\end{Shaded}

The exponentiated coefficients of the second model are not very
different from the first model:

\begin{itemize}
\tightlist
\item
  All else equal, a unit change in Elevation \emph{reduces} the chances
  of flood inundation by 46.4\%
\item
  All else equal, a unit change in Flow Accumulation \emph{increases}
  the odds of flood inundation by 10.6\%
\item
  All else equal, a unit change in Distance to Stream \emph{reduces} the
  odds of flood inundation by 40.5\%
\end{itemize}

\begin{Shaded}
\begin{Highlighting}[]
\DocumentationTok{\#\# \% change in Y for unit change in X = [(exponent (coefficient of X) {-} 1)] * 100}
\DocumentationTok{\#\# {-} or + sign indicates associated increase or decrease}

\NormalTok{inundModel\_2}\SpecialCharTok{$}\NormalTok{coefficients}

\NormalTok{inundModel2\_vars }\OtherTok{\textless{}{-}} \FunctionTok{c}\NormalTok{(}\StringTok{"Elevation"}\NormalTok{, }\StringTok{"Flow Accumulation (log)"}\NormalTok{, }\StringTok{"Distance to Stream (log)"}\NormalTok{)}
\NormalTok{inundModel2\_coeffs }\OtherTok{\textless{}{-}} \FunctionTok{c}\NormalTok{(((}\FunctionTok{exp}\NormalTok{(}\SpecialCharTok{{-}}\FloatTok{0.6243868}\NormalTok{) }\SpecialCharTok{{-}} \DecValTok{1}\NormalTok{) }\SpecialCharTok{*} \DecValTok{100}\NormalTok{), ((}\FunctionTok{exp}\NormalTok{(}\FloatTok{0.1009589}\NormalTok{) }\SpecialCharTok{{-}} \DecValTok{1}\NormalTok{) }\SpecialCharTok{*} \DecValTok{100}\NormalTok{), ((}\FunctionTok{exp}\NormalTok{(}\SpecialCharTok{{-}}\FloatTok{0.5192783}\NormalTok{) }\SpecialCharTok{{-}} \DecValTok{1}\NormalTok{) }\SpecialCharTok{*} \DecValTok{100}\NormalTok{))}

\NormalTok{inundModel2\_coefficients }\OtherTok{\textless{}{-}} \FunctionTok{data.frame}\NormalTok{(inundModel2\_vars, inundModel2\_coeffs)}

\NormalTok{inundModel2\_coefficients }\SpecialCharTok{\%\textgreater{}\%} 
  \FunctionTok{kbl}\NormalTok{(}\AttributeTok{caption =} \StringTok{"Exponentiated Coefficients: Logistic Regression Model 2"}\NormalTok{) }\SpecialCharTok{\%\textgreater{}\%} 
   \FunctionTok{kable\_styling}\NormalTok{(}\AttributeTok{bootstrap\_options =} \StringTok{"striped"}\NormalTok{, }\AttributeTok{full\_width =}\NormalTok{ F, }\AttributeTok{position =} \StringTok{"left"}\NormalTok{)}
\end{Highlighting}
\end{Shaded}

\textbf{Model 3: What if we don't log-adjust our variables?}

While carrying out a preliminary assessment of variable relationships,
we log-adjusted Flow Accumulation and Distance to Stream to compress
their range of values.

But what happens if we use the non-adjusted variables in a logit model?

\begin{itemize}
\item
  All variables except Land Porosity are statistically significant.
\item
  The AIC is lower than the previous two models at \textbf{1185.2}
  (compared to 1253.8 and 1252.2 for models 1 and 2 respectively).
\end{itemize}

Does it mess with our coefficient interpretation?

\begin{Shaded}
\begin{Highlighting}[]
\NormalTok{inundTrain\_3 }\OtherTok{\textless{}{-}}\NormalTok{ calg\_dat[ trainIndex,] }\SpecialCharTok{\%\textgreater{}\%} 
\NormalTok{  dplyr}\SpecialCharTok{::}\FunctionTok{select}\NormalTok{(}\SpecialCharTok{{-}}\NormalTok{flowac\_mean\_log, }\SpecialCharTok{{-}}\NormalTok{streamdist\_min\_log)}

\NormalTok{inundTest\_3  }\OtherTok{\textless{}{-}}\NormalTok{ calg\_dat[}\SpecialCharTok{{-}}\NormalTok{trainIndex,]}\SpecialCharTok{\%\textgreater{}\%} 
\NormalTok{  dplyr}\SpecialCharTok{::}\FunctionTok{select}\NormalTok{(}\SpecialCharTok{{-}}\NormalTok{flowac\_mean\_log, }\SpecialCharTok{{-}}\NormalTok{streamdist\_min\_log)}


\NormalTok{inundModel\_3 }\OtherTok{\textless{}{-}} \FunctionTok{glm}\NormalTok{(inund\_sum }\SpecialCharTok{\textasciitilde{}}\NormalTok{ ., }
                    \AttributeTok{family=}\StringTok{"binomial"}\NormalTok{(}\AttributeTok{link=}\StringTok{"logit"}\NormalTok{), }\AttributeTok{data =}\NormalTok{ inundTrain\_3 }\SpecialCharTok{\%\textgreater{}\%}
                                                            \FunctionTok{as.data.frame}\NormalTok{() }\SpecialCharTok{\%\textgreater{}\%}
\NormalTok{                                                            dplyr}\SpecialCharTok{::}\FunctionTok{select}\NormalTok{(}\SpecialCharTok{{-}}\NormalTok{geometry, }\SpecialCharTok{{-}}\NormalTok{uniqueID))}
\FunctionTok{summary}\NormalTok{(inundModel\_3)}
\end{Highlighting}
\end{Shaded}

The exponentiated coefficients of the third model are very different
from the previous two models, in that the odds of inundation associated
with Land Porosity, non-log-adjusted Flow Accumulation and Distance to
Streams is a \emph{lot} smaller:

\begin{itemize}
\tightlist
\item
  All else equal, a unit change in Land Porosity \emph{increases} the
  chances of flood inundation by 4.9\%
\item
  All else equal, a unit change in Elevation \emph{reduces} the chances
  of flood inundation by 38.1\%
\item
  All else equal, a unit change in Flow Accumulation \emph{increases}
  the odds of flood inundation by 0.01\%
\item
  All else equal, a unit change in Distance to Stream \emph{reduces} the
  odds of flood inundation by 0.46\%
\end{itemize}

This seems funky, given that a variation in distance from a stream
should logically affect the odds of inundation a lot more than 0.46\%.
Also, since this project is less about statistical significance and more
about exploring variables, perhaps including Land Porosity would be fun.

We're glad (lol) we ran these options, but we're going to stick with our
original model for the rest of the project ¯\_(ツ)\_/¯

\begin{Shaded}
\begin{Highlighting}[]
\DocumentationTok{\#\# \% change in Y for unit change in X = [(exponent (coefficient of X) {-} 1)] * 100}
\DocumentationTok{\#\# {-} or + sign indicates associated increase or decrease}

\NormalTok{inundModel\_3}\SpecialCharTok{$}\NormalTok{coefficients}

\NormalTok{inundModel3\_vars }\OtherTok{\textless{}{-}} \FunctionTok{c}\NormalTok{(}\StringTok{"Land Porosity"}\NormalTok{, }\StringTok{"Elevation"}\NormalTok{, }\StringTok{"Flow Accumulation"}\NormalTok{, }\StringTok{"Distance to Stream"}\NormalTok{)}
\NormalTok{inundModel3\_coeffs }\OtherTok{\textless{}{-}} \FunctionTok{c}\NormalTok{(((}\FunctionTok{exp}\NormalTok{(}\FloatTok{0.0479009901}\NormalTok{) }\SpecialCharTok{{-}} \DecValTok{1}\NormalTok{) }\SpecialCharTok{*} \DecValTok{100}\NormalTok{), ((}\FunctionTok{exp}\NormalTok{(}\SpecialCharTok{{-}}\FloatTok{0.4802761173}\NormalTok{) }\SpecialCharTok{{-}} \DecValTok{1}\NormalTok{) }\SpecialCharTok{*} \DecValTok{100}\NormalTok{), ((}\FunctionTok{exp}\NormalTok{(}\FloatTok{0.0001509916}\NormalTok{) }\SpecialCharTok{{-}} \DecValTok{1}\NormalTok{) }\SpecialCharTok{*} \DecValTok{100}\NormalTok{), ((}\FunctionTok{exp}\NormalTok{(}\SpecialCharTok{{-}}\FloatTok{0.0046012831}\NormalTok{) }\SpecialCharTok{{-}} \DecValTok{1}\NormalTok{) }\SpecialCharTok{*} \DecValTok{100}\NormalTok{))}

\NormalTok{inundModel3\_coefficients }\OtherTok{\textless{}{-}} \FunctionTok{data.frame}\NormalTok{(inundModel3\_vars, inundModel3\_coeffs)}

\NormalTok{inundModel3\_coefficients }\SpecialCharTok{\%\textgreater{}\%} 
  \FunctionTok{kbl}\NormalTok{(}\AttributeTok{caption =} \StringTok{"Exponentiated Coefficients: Logistic Regression Model 2"}\NormalTok{) }\SpecialCharTok{\%\textgreater{}\%} 
   \FunctionTok{kable\_styling}\NormalTok{(}\AttributeTok{bootstrap\_options =} \StringTok{"striped"}\NormalTok{, }\AttributeTok{full\_width =}\NormalTok{ F, }\AttributeTok{position =} \StringTok{"left"}\NormalTok{)}
\end{Highlighting}
\end{Shaded}
\end{block}

\begin{block}{3.3 Model Validation}
\protect\hypertarget{model-validation}{}
\textbf{So, Is My House Going to Flood?}

Well, it depends on where you live.

The plots below illustrate a distribution of predicted probabilities,
based on the training and test sets.

\begin{itemize}
\item
  The histogram illustrates the predicted probability of fishnet cells
  in Calgary being inundated, based on the conditions set by our first
  model. As the frequency distribution shows, the dataset exhibits a
  lower overall probability of inundation -- there are over 1200 fishnet
  cells with a 0-0.1 probability of inundation, whereas cells with a
  higher probability of inundation are much fewer.
\item
  The second plot is a measure of how well our data is predicting
  probabilities for ``inundation'' (1s) vs ``no inundation'' (0s), where
  the vertical line represents a 0.5 probability of inundation. In this
  case the values reflecting not inundated areas are clustered closer to
  zero, indicating a lower probability for inundation overall.
\end{itemize}

\begin{Shaded}
\begin{Highlighting}[]
\NormalTok{classProbs }\OtherTok{\textless{}{-}} \FunctionTok{predict}\NormalTok{(inundModel, inundTest, }\AttributeTok{type=}\StringTok{"response"}\NormalTok{)}


\FunctionTok{hist}\NormalTok{(classProbs)}
\end{Highlighting}
\end{Shaded}

\includegraphics{luem_a3_markdown_presentation_files/figure-beamer/predict_first-1.pdf}

\begin{Shaded}
\begin{Highlighting}[]
\DocumentationTok{\#\#histogram is for the whole dataset}
\DocumentationTok{\#\#represents the probability that a cell will be inundated (x{-}axis), vs number of cells with that probability (y{-}axis)}
\end{Highlighting}
\end{Shaded}

\begin{Shaded}
\begin{Highlighting}[]
\NormalTok{testProbs }\OtherTok{\textless{}{-}} \FunctionTok{data.frame}\NormalTok{(}\AttributeTok{obs =} \FunctionTok{as.numeric}\NormalTok{(inundTest}\SpecialCharTok{$}\NormalTok{inund\_sum),}
                        \AttributeTok{pred =}\NormalTok{ classProbs)}

\FunctionTok{ggplot}\NormalTok{(testProbs, }\FunctionTok{aes}\NormalTok{(}\AttributeTok{x =}\NormalTok{ pred, }\AttributeTok{fill=}\FunctionTok{as.factor}\NormalTok{(obs))) }\SpecialCharTok{+} 
  \FunctionTok{geom\_density}\NormalTok{() }\SpecialCharTok{+}
  \FunctionTok{facet\_grid}\NormalTok{(obs }\SpecialCharTok{\textasciitilde{}}\NormalTok{ .) }\SpecialCharTok{+} 
  \FunctionTok{xlab}\NormalTok{(}\StringTok{"Probability"}\NormalTok{) }\SpecialCharTok{+} 
  \FunctionTok{ylab}\NormalTok{(}\StringTok{"Frequency"}\NormalTok{) }\SpecialCharTok{+}
  \FunctionTok{geom\_vline}\NormalTok{(}\AttributeTok{xintercept =}\NormalTok{ .}\DecValTok{5}\NormalTok{) }\SpecialCharTok{+}
  \FunctionTok{scale\_fill\_manual}\NormalTok{(}\AttributeTok{values =} \FunctionTok{c}\NormalTok{(}\StringTok{"\#CEEBF0"}\NormalTok{, }\StringTok{"\#51A6AE"}\NormalTok{),}
                      \AttributeTok{labels =} \FunctionTok{c}\NormalTok{(}\StringTok{"Not Inundated"}\NormalTok{,}\StringTok{"Inundated"}\NormalTok{)) }\SpecialCharTok{+}
  \FunctionTok{labs}\NormalTok{(}\AttributeTok{title =} \StringTok{"Number of Fishnet Cells Associated with Flooding in Calgary"}\NormalTok{)}\SpecialCharTok{+}
\NormalTok{  plotTheme}
\end{Highlighting}
\end{Shaded}

\includegraphics{luem_a3_markdown_presentation_files/figure-beamer/plot_preds-1.pdf}
\end{block}
\end{frame}

\begin{frame}[fragile]{4. Confusion, Indeed}
\protect\hypertarget{confusion-indeed}{}
\begin{block}{4.1 Confusion Metrics}
\protect\hypertarget{confusion-metrics}{}
To test the model's prediction accuracy, we created a confusion matrix
from which levels of error can be extrapolated. If we assume a
probability cutoff threshold of 50\%, our confusion matrix gives us an
accuracy of \textbf{0.909}, with the following sensitivity and
specificity results:

\textbf{Sensitivity \& Specificity Analysis}: Predicted vs.~Reference
Values, 50\% Cutoff

\begin{itemize}
\tightlist
\item
  It has predicted 0 as 0, i.e.~a \emph{True Negative}, 1389 times
\item
  It has predicted 0 as 1, i.e.~a \emph{False Negative}, 91 times
\item
  It has predicted 1 as 0, i.e.~a \emph{False Positive}, 55 times
\item
  It has predicted 1 as 1, i.e.~a \emph{True Positive}, 75 times
\end{itemize}

Overall, the model's true positive rate, i.e., the proportion of 1s
predicted as 1s indicates its \emph{sensitivity}, which in this case is
almost twice as much as the number of 1s falsely predicted as 0s. The
model's true negative rate, i.e., the proportion of 0s accurately
predicted as 0s is about 15 times greater than the number of 0s falsely
predicted as 1s, indicating the model's \emph{specificity}.

It has an error rate of about 10\% as derived from the accuracy value
(Error = Accuracy - 1), which is demonstrated by the fact that the
model's 146 erroneous predictions are only a small fraction of its 1464
accurate predictions.

\begin{Shaded}
\begin{Highlighting}[]
\NormalTok{testProbs}\SpecialCharTok{$}\NormalTok{predClass  }\OtherTok{=} \FunctionTok{ifelse}\NormalTok{(testProbs}\SpecialCharTok{$}\NormalTok{pred }\SpecialCharTok{\textgreater{}}\NormalTok{ .}\DecValTok{5}\NormalTok{ ,}\DecValTok{1}\NormalTok{,}\DecValTok{0}\NormalTok{)}

\NormalTok{caret}\SpecialCharTok{::}\FunctionTok{confusionMatrix}\NormalTok{(}\AttributeTok{reference =} \FunctionTok{as.factor}\NormalTok{(testProbs}\SpecialCharTok{$}\NormalTok{obs), }
                       \AttributeTok{data =} \FunctionTok{as.factor}\NormalTok{(testProbs}\SpecialCharTok{$}\NormalTok{predClass), }
                       \AttributeTok{positive =} \StringTok{"1"}\NormalTok{)}
\end{Highlighting}
\end{Shaded}

\begin{verbatim}
## Confusion Matrix and Statistics
## 
##           Reference
## Prediction    0    1
##          0 1389   91
##          1   55   75
##                                           
##                Accuracy : 0.9093          
##                  95% CI : (0.8942, 0.9229)
##     No Information Rate : 0.8969          
##     P-Value [Acc > NIR] : 0.053027        
##                                           
##                   Kappa : 0.4576          
##                                           
##  Mcnemar's Test P-Value : 0.003772        
##                                           
##             Sensitivity : 0.45181         
##             Specificity : 0.96191         
##          Pos Pred Value : 0.57692         
##          Neg Pred Value : 0.93851         
##              Prevalence : 0.10311         
##          Detection Rate : 0.04658         
##    Detection Prevalence : 0.08075         
##       Balanced Accuracy : 0.70686         
##                                           
##        'Positive' Class : 1               
## 
\end{verbatim}

We experimented with the threshold cutoff to check how the model would
respond, and found that:

The accuracy of a model with with a \textbf{75\%} threshold cutoff is a
little higher than the original model with a 50\% cutoff, at
\textbf{0.917}. However, it contains slightly different levels of False
Positive and False Negative predictions.

Similarly, the accuracy of a model with a \textbf{25\%} cutoff is only
marginally higher than the original model at \textbf{0.903}, but it also
exhibits a lower specificity.

\begin{Shaded}
\begin{Highlighting}[]
\NormalTok{testProbs}\SpecialCharTok{$}\NormalTok{predClass75  }\OtherTok{=} \FunctionTok{ifelse}\NormalTok{(testProbs}\SpecialCharTok{$}\NormalTok{pred }\SpecialCharTok{\textgreater{}}\NormalTok{ .}\DecValTok{75}\NormalTok{ ,}\DecValTok{1}\NormalTok{,}\DecValTok{0}\NormalTok{)}

\NormalTok{caret}\SpecialCharTok{::}\FunctionTok{confusionMatrix}\NormalTok{(}\AttributeTok{reference =} \FunctionTok{as.factor}\NormalTok{(testProbs}\SpecialCharTok{$}\NormalTok{obs), }
                       \AttributeTok{data =} \FunctionTok{as.factor}\NormalTok{(testProbs}\SpecialCharTok{$}\NormalTok{predClass75), }
                       \AttributeTok{positive =} \StringTok{"1"}\NormalTok{)}
\end{Highlighting}
\end{Shaded}

\begin{verbatim}
## Confusion Matrix and Statistics
## 
##           Reference
## Prediction    0    1
##          0 1440  129
##          1    4   37
##                                                
##                Accuracy : 0.9174               
##                  95% CI : (0.9029, 0.9304)     
##     No Information Rate : 0.8969               
##     P-Value [Acc > NIR] : 0.003105             
##                                                
##                   Kappa : 0.3301               
##                                                
##  Mcnemar's Test P-Value : < 0.00000000000000022
##                                                
##             Sensitivity : 0.22289              
##             Specificity : 0.99723              
##          Pos Pred Value : 0.90244              
##          Neg Pred Value : 0.91778              
##              Prevalence : 0.10311              
##          Detection Rate : 0.02298              
##    Detection Prevalence : 0.02547              
##       Balanced Accuracy : 0.61006              
##                                                
##        'Positive' Class : 1                    
## 
\end{verbatim}

\begin{Shaded}
\begin{Highlighting}[]
\NormalTok{testProbs}\SpecialCharTok{$}\NormalTok{predClass25  }\OtherTok{=} \FunctionTok{ifelse}\NormalTok{(testProbs}\SpecialCharTok{$}\NormalTok{pred }\SpecialCharTok{\textgreater{}}\NormalTok{ .}\DecValTok{25}\NormalTok{ ,}\DecValTok{1}\NormalTok{,}\DecValTok{0}\NormalTok{)}

\NormalTok{caret}\SpecialCharTok{::}\FunctionTok{confusionMatrix}\NormalTok{(}\AttributeTok{reference =} \FunctionTok{as.factor}\NormalTok{(testProbs}\SpecialCharTok{$}\NormalTok{obs), }
                       \AttributeTok{data =} \FunctionTok{as.factor}\NormalTok{(testProbs}\SpecialCharTok{$}\NormalTok{predClass25), }
                       \AttributeTok{positive =} \StringTok{"1"}\NormalTok{)}
\end{Highlighting}
\end{Shaded}

\begin{verbatim}
## Confusion Matrix and Statistics
## 
##           Reference
## Prediction    0    1
##          0 1342   53
##          1  102  113
##                                           
##                Accuracy : 0.9037          
##                  95% CI : (0.8883, 0.9177)
##     No Information Rate : 0.8969          
##     P-Value [Acc > NIR] : 0.1955061       
##                                           
##                   Kappa : 0.5396          
##                                           
##  Mcnemar's Test P-Value : 0.0001155       
##                                           
##             Sensitivity : 0.68072         
##             Specificity : 0.92936         
##          Pos Pred Value : 0.52558         
##          Neg Pred Value : 0.96201         
##              Prevalence : 0.10311         
##          Detection Rate : 0.07019         
##    Detection Prevalence : 0.13354         
##       Balanced Accuracy : 0.80504         
##                                           
##        'Positive' Class : 1               
## 
\end{verbatim}
\end{block}

\begin{block}{4.2 ROC Curve}
\protect\hypertarget{roc-curve}{}
\textbf{Run, These Are No Geese!!}

The Receiver Operating Characteristic (ROC) curve for our original model
with a 50\% cutoff plots the True Positive Rate (Sensitivity) against
the False Positive Rate (1 - Specificity) for the probability of flood
inundation in Calgary.

It shows the trade-off between Sensitivity and Specificity in our test
set against a random classifier where True Positive = False Positive
(the light grey diagonal). In this case, the variation of the ROC curve
away from the random classifier line, toward the top-right corner
indicates a high rate of model accuracy.

This is further confirmed by calculating the area under the curve as
\textbf{0.939}, which confirms that the model is able to predict flood
inundation with a 94\% accuracy.

\begin{Shaded}
\begin{Highlighting}[]
\FunctionTok{ggplot}\NormalTok{(testProbs, }\FunctionTok{aes}\NormalTok{(}\AttributeTok{d =}\NormalTok{ obs, }\AttributeTok{m =}\NormalTok{ pred)) }\SpecialCharTok{+} 
  \FunctionTok{geom\_roc}\NormalTok{(}\AttributeTok{n.cuts =} \DecValTok{50}\NormalTok{, }\AttributeTok{labels =} \ConstantTok{FALSE}\NormalTok{) }\SpecialCharTok{+} 
  \FunctionTok{style\_roc}\NormalTok{(}\AttributeTok{theme =}\NormalTok{ theme\_grey) }\SpecialCharTok{+}
  \FunctionTok{geom\_abline}\NormalTok{(}\AttributeTok{slope =} \DecValTok{1}\NormalTok{, }\AttributeTok{intercept =} \DecValTok{0}\NormalTok{, }\AttributeTok{size =} \FloatTok{1.5}\NormalTok{, }\AttributeTok{color =} \StringTok{\textquotesingle{}grey\textquotesingle{}}\NormalTok{) }
\end{Highlighting}
\end{Shaded}

\includegraphics{luem_a3_markdown_presentation_files/figure-beamer/roc_curve-1.pdf}

\begin{Shaded}
\begin{Highlighting}[]
\FunctionTok{auc}\NormalTok{(testProbs}\SpecialCharTok{$}\NormalTok{obs, testProbs}\SpecialCharTok{$}\NormalTok{pred)}
\end{Highlighting}
\end{Shaded}

\begin{verbatim}
## Area under the curve: 0.9398
\end{verbatim}
\end{block}

\begin{block}{4.3 Cross Validation}
\protect\hypertarget{cross-validation}{}
\textbf{It's Accurate but is it Generalizable?}

The following section checks the accuracy of our predictions across 100
randomly generated test sets, to gauge its applicability in predicting
Denver's chances of flood inundation.

On average, the prediction accuracy across all 100 folds is
\textbf{92\%}.

\begin{Shaded}
\begin{Highlighting}[]
\NormalTok{ctrl }\OtherTok{\textless{}{-}} \FunctionTok{trainControl}\NormalTok{(}\AttributeTok{method =} \StringTok{"cv"}\NormalTok{, }
                     \AttributeTok{number =} \DecValTok{100}\NormalTok{, }
                     \AttributeTok{savePredictions =} \ConstantTok{TRUE}\NormalTok{)}

\NormalTok{inundFit }\OtherTok{\textless{}{-}} \FunctionTok{train}\NormalTok{(}\FunctionTok{as.factor}\NormalTok{(inund\_sum) }\SpecialCharTok{\textasciitilde{}}\NormalTok{ .,}
               \AttributeTok{data =}\NormalTok{ calg\_dat }\SpecialCharTok{\%\textgreater{}\%} 
                 \FunctionTok{as.data.frame}\NormalTok{() }\SpecialCharTok{\%\textgreater{}\%}
\NormalTok{                 dplyr}\SpecialCharTok{::}\FunctionTok{select}\NormalTok{(inund\_sum, pervious\_mean, elevation\_mean, flowac\_mean\_log, streamdist\_min\_log), }
               \AttributeTok{method=}\StringTok{"glm"}\NormalTok{, }\AttributeTok{family=}\StringTok{"binomial"}\NormalTok{,}
               \AttributeTok{trControl =}\NormalTok{ ctrl)}

\NormalTok{inundFit}
\end{Highlighting}
\end{Shaded}

\begin{verbatim}
## Generalized Linear Model 
## 
## 5373 samples
##    4 predictor
##    2 classes: '0', '1' 
## 
## No pre-processing
## Resampling: Cross-Validated (100 fold) 
## Summary of sample sizes: 5319, 5319, 5319, 5319, 5318, 5318, ... 
## Resampling results:
## 
##   Accuracy   Kappa    
##   0.9184835  0.4662349
\end{verbatim}

\begin{Shaded}
\begin{Highlighting}[]
\CommentTok{\#inundFit is our model trained to predict using the binomial logistic regression, or glm, method. }
\end{Highlighting}
\end{Shaded}

Subsequently plotting a histogram of the accuracy for each fold allows
us to trace its generalizability. The plot points out that a large
number of folds (i.e., values from the ``Resample'' column) are
clustered at high accuracy values. This indicates a high level of
generalizability -- a measure of the model's capacity to be applied to
predict other sample sets -- in our case, flood inundation in Denver.

It gives the model confidence in moving to the final stage of the
project -- applying Calgary's flood inundation predictions to the city
of Denver.

\begin{Shaded}
\begin{Highlighting}[]
\FunctionTok{ggplot}\NormalTok{(}\FunctionTok{as.data.frame}\NormalTok{(inundFit}\SpecialCharTok{$}\NormalTok{resample), }\FunctionTok{aes}\NormalTok{(Accuracy)) }\SpecialCharTok{+} 
  \FunctionTok{geom\_histogram}\NormalTok{() }\SpecialCharTok{+}
  \FunctionTok{scale\_x\_continuous}\NormalTok{(}\AttributeTok{limits =} \FunctionTok{c}\NormalTok{(}\DecValTok{0}\NormalTok{, }\DecValTok{1}\NormalTok{)) }\SpecialCharTok{+}
  \FunctionTok{labs}\NormalTok{(}\AttributeTok{x=}\StringTok{"Accuracy"}\NormalTok{,}
       \AttributeTok{y=}\StringTok{"Count"}\NormalTok{)}\SpecialCharTok{+}
\NormalTok{  plotTheme}
\end{Highlighting}
\end{Shaded}

\includegraphics{luem_a3_markdown_presentation_files/figure-beamer/cv_hist-1.pdf}
\end{block}
\end{frame}

\begin{frame}[fragile]{5. Map Predictions}
\protect\hypertarget{map-predictions}{}
\textbf{The Moment We've Been Waiting For}

We ran a lot of tests on our model, and are probably ready to use it for
a few predictions, now.

\begin{block}{5.1 Predictions for Calgary}
\protect\hypertarget{predictions-for-calgary}{}
\begin{Shaded}
\begin{Highlighting}[]
\NormalTok{calg\_dat\_log }\OtherTok{\textless{}{-}}\NormalTok{ calg\_dat }\SpecialCharTok{\%\textgreater{}\%} 
\NormalTok{  dplyr}\SpecialCharTok{::}\FunctionTok{select}\NormalTok{(uniqueID, inund\_sum, pervious\_mean, elevation\_mean, flowac\_mean\_log, streamdist\_min\_log, geometry)}


\NormalTok{allPredictions }\OtherTok{\textless{}{-}} 
  \FunctionTok{predict}\NormalTok{(inundFit, calg\_dat, }\AttributeTok{type=}\StringTok{"prob"}\NormalTok{)[,}\DecValTok{2}\NormalTok{]}
  
\NormalTok{calg\_pred }\OtherTok{\textless{}{-}} 
  \FunctionTok{cbind}\NormalTok{(calg\_dat\_log,allPredictions) }\SpecialCharTok{\%\textgreater{}\%}
  \FunctionTok{mutate}\NormalTok{(}\AttributeTok{allPredictions =} \FunctionTok{round}\NormalTok{(allPredictions }\SpecialCharTok{*} \DecValTok{100}\NormalTok{)) }
\end{Highlighting}
\end{Shaded}

\begin{Shaded}
\begin{Highlighting}[]
 \FunctionTok{ggplot}\NormalTok{() }\SpecialCharTok{+} 
    \FunctionTok{geom\_sf}\NormalTok{(}\AttributeTok{data=}\NormalTok{calg\_pred, }\FunctionTok{aes}\NormalTok{(}\AttributeTok{fill=}\FunctionTok{factor}\NormalTok{(}\FunctionTok{ntile}\NormalTok{(allPredictions,}\DecValTok{4}\NormalTok{))), }
            \AttributeTok{colour=}\ConstantTok{NA}\NormalTok{) }\SpecialCharTok{+}
    \FunctionTok{scale\_fill\_manual}\NormalTok{(}\AttributeTok{values =}\NormalTok{ blues,}
                      \AttributeTok{labels=}\FunctionTok{as.character}\NormalTok{(}\FunctionTok{quantile}\NormalTok{(calg\_pred}\SpecialCharTok{$}\NormalTok{allPredictions,}
                                                 \FunctionTok{c}\NormalTok{(}\FloatTok{0.2}\NormalTok{,.}\DecValTok{4}\NormalTok{,.}\DecValTok{6}\NormalTok{,.}\DecValTok{8}\NormalTok{),}
                                                 \AttributeTok{na.rm=}\NormalTok{T)),}
                      \AttributeTok{name=}\StringTok{"Predicted}\SpecialCharTok{\textbackslash{}n}\StringTok{Probabilities(\%)}\SpecialCharTok{\textbackslash{}n}\StringTok{(Grouped in}\SpecialCharTok{\textbackslash{}n}\StringTok{Quantile}\SpecialCharTok{\textbackslash{}n}\StringTok{Breaks)"}\NormalTok{) }\SpecialCharTok{+}
\NormalTok{  mapTheme }\SpecialCharTok{+}
  \FunctionTok{labs}\NormalTok{(}\AttributeTok{title=}\StringTok{"Predicted Probability of Flood Inundation in Calgary"}\NormalTok{,}
       \AttributeTok{subtitle =} \StringTok{"Based on a Logistic Regression Model"}\NormalTok{)}
\end{Highlighting}
\end{Shaded}

\includegraphics{luem_a3_markdown_presentation_files/figure-beamer/predicted_map1-1.pdf}

Let's map it again with the already calculated inundation types
overlaid.

\begin{Shaded}
\begin{Highlighting}[]
 \FunctionTok{ggplot}\NormalTok{() }\SpecialCharTok{+} 
  \FunctionTok{geom\_sf}\NormalTok{(}\AttributeTok{data=}\NormalTok{calg\_pred, }\FunctionTok{aes}\NormalTok{(}\AttributeTok{fill=}\FunctionTok{factor}\NormalTok{(}\FunctionTok{ntile}\NormalTok{(allPredictions,}\DecValTok{4}\NormalTok{))), }\AttributeTok{colour=}\ConstantTok{NA}\NormalTok{) }\SpecialCharTok{+}
  \FunctionTok{scale\_fill\_manual}\NormalTok{(}\AttributeTok{values =}\NormalTok{ blues,}
                    \AttributeTok{labels=}\FunctionTok{as.character}\NormalTok{(}\FunctionTok{quantile}\NormalTok{(calg\_pred}\SpecialCharTok{$}\NormalTok{allPredictions,}
                                                 \FunctionTok{c}\NormalTok{(.}\DecValTok{2}\NormalTok{,.}\DecValTok{4}\NormalTok{,.}\DecValTok{6}\NormalTok{,.}\DecValTok{8}\NormalTok{),}
                                                 \AttributeTok{na.rm=}\NormalTok{T)),}
                    \AttributeTok{name=}\StringTok{"Predicted}\SpecialCharTok{\textbackslash{}n}\StringTok{Probabilities(\%)}\SpecialCharTok{\textbackslash{}n}\StringTok{(Grouped in}\SpecialCharTok{\textbackslash{}n}\StringTok{Quintile}\SpecialCharTok{\textbackslash{}n}\StringTok{Breaks)"}\NormalTok{) }\SpecialCharTok{+}
  \FunctionTok{geom\_sf}\NormalTok{(}\AttributeTok{data=}\NormalTok{calg\_pred  }\SpecialCharTok{\%\textgreater{}\%} 
               \FunctionTok{filter}\NormalTok{(inund\_sum }\SpecialCharTok{==} \DecValTok{1}\NormalTok{), }
               \AttributeTok{fill=}\StringTok{"\#EDBA46"}\NormalTok{, }\AttributeTok{alpha=}\FloatTok{0.9}\NormalTok{, }\AttributeTok{colour=}\ConstantTok{NA}\NormalTok{) }\SpecialCharTok{+}
    \FunctionTok{geom\_sf}\NormalTok{(}\AttributeTok{data=}\NormalTok{calg\_pred }\SpecialCharTok{\%\textgreater{}\%} 
              \FunctionTok{filter}\NormalTok{(inund\_sum }\SpecialCharTok{==} \DecValTok{0}\NormalTok{), }
            \AttributeTok{fill=}\StringTok{"\#F9E2B2"}\NormalTok{, }\AttributeTok{alpha=}\FloatTok{0.35}\NormalTok{,}\AttributeTok{colour=}\ConstantTok{NA}\NormalTok{) }\SpecialCharTok{+}  
\NormalTok{  mapTheme }\SpecialCharTok{+}
  \FunctionTok{labs}\NormalTok{(}\AttributeTok{title=}\StringTok{"Observed and Predicted Flood Inundation Areas"}\NormalTok{,}
       \AttributeTok{subtitle=}\StringTok{"Yellow marks areas with observed \textquotesingle{}inundation\textquotesingle{}, }\SpecialCharTok{\textbackslash{}n}\StringTok{all other taken as \textquotesingle{}not inundated\textquotesingle{} for the purpose of binary regression modeling"}\NormalTok{)}
\end{Highlighting}
\end{Shaded}

\includegraphics{luem_a3_markdown_presentation_files/figure-beamer/predicted_map2-1.pdf}

Looks like our predicted and observed values are very close!

How are the Sensitivity and Specificity values laid out on the map of
Calgary?

\begin{Shaded}
\begin{Highlighting}[]
\NormalTok{calg\_pred }\SpecialCharTok{\%\textgreater{}\%}
  \FunctionTok{mutate}\NormalTok{(}\AttributeTok{confResult=}\FunctionTok{case\_when}\NormalTok{(allPredictions }\SpecialCharTok{\textless{}} \DecValTok{50} \SpecialCharTok{\&}\NormalTok{ inund\_sum}\SpecialCharTok{==}\DecValTok{0} \SpecialCharTok{\textasciitilde{}} \StringTok{"True Negative"}\NormalTok{,}
\NormalTok{                              allPredictions }\SpecialCharTok{\textgreater{}=} \DecValTok{50} \SpecialCharTok{\&}\NormalTok{ inund\_sum}\SpecialCharTok{==}\DecValTok{1} \SpecialCharTok{\textasciitilde{}} \StringTok{"True Positive"}\NormalTok{,}
\NormalTok{                              allPredictions }\SpecialCharTok{\textless{}} \DecValTok{50} \SpecialCharTok{\&}\NormalTok{ inund\_sum}\SpecialCharTok{==}\DecValTok{1} \SpecialCharTok{\textasciitilde{}} \StringTok{"False Negative"}\NormalTok{,}
\NormalTok{                              allPredictions }\SpecialCharTok{\textgreater{}=} \DecValTok{50} \SpecialCharTok{\&}\NormalTok{ inund\_sum}\SpecialCharTok{==}\DecValTok{0} \SpecialCharTok{\textasciitilde{}} \StringTok{"False Positive"}\NormalTok{)) }\SpecialCharTok{\%\textgreater{}\%}
  \FunctionTok{ggplot}\NormalTok{()}\SpecialCharTok{+}
  \FunctionTok{geom\_sf}\NormalTok{(}\FunctionTok{aes}\NormalTok{(}\AttributeTok{fill =}\NormalTok{ confResult), }\AttributeTok{color =} \StringTok{"transparent"}\NormalTok{)}\SpecialCharTok{+}
  \FunctionTok{scale\_fill\_manual}\NormalTok{(}\AttributeTok{values =} \FunctionTok{c}\NormalTok{(}\StringTok{"\#B0B0B3"}\NormalTok{,}\StringTok{"\#A2D3D8"}\NormalTok{,}\StringTok{"\#FFEFDE"}\NormalTok{,}\StringTok{"\#81996F"}\NormalTok{),}
                    \AttributeTok{name=}\StringTok{"Outcomes"}\NormalTok{)}\SpecialCharTok{+}
  \FunctionTok{labs}\NormalTok{(}\AttributeTok{title=}\StringTok{"Confusion Metrics"}\NormalTok{) }\SpecialCharTok{+}
\NormalTok{  mapTheme}
\end{Highlighting}
\end{Shaded}

\includegraphics{luem_a3_markdown_presentation_files/figure-beamer/error_map-1.pdf}

In spite of the confusion matrix indicating low levels of false
negatives proportionate to the data we processed for Calgary, the map
helps spatially visualize how the model's errors are spread out, and
what might be contributing to them:

\begin{itemize}
\tightlist
\item
  In pre-processing our data, we reclassified the Land Cover rasters for
  both cities to a 0/1 binary to indicate Land Porosity, where we kept
  urbanized and rocky land as ``non-pervious'', whereas all natural
  vegetation and marshes were ``pervious''. This assumption could have
  reduced the model's accuracy by reducing continuous run-off potential
  numbers to binary, thus reducing nuances in the relationship between
  Land Porosity and Inundation that the model could capture.
\end{itemize}
\end{block}

\begin{block}{5.2 Predictions for Denver}
\protect\hypertarget{predictions-for-denver}{}
\begin{Shaded}
\begin{Highlighting}[]
\NormalTok{denver\_dat\_log }\OtherTok{\textless{}{-}}\NormalTok{ denver\_dat }\SpecialCharTok{\%\textgreater{}\%} 
\NormalTok{  dplyr}\SpecialCharTok{::}\FunctionTok{select}\NormalTok{(uniqueID, pervious\_mean, elevation\_mean, flowac\_mean\_log, streamdist\_min\_log, geometry)}

\NormalTok{allPredictions\_denver }\OtherTok{\textless{}{-}} 
  \FunctionTok{predict}\NormalTok{(inundFit, denver\_dat\_log, }\AttributeTok{type=}\StringTok{"prob"}\NormalTok{)[,}\DecValTok{2}\NormalTok{]}
  
\NormalTok{denver\_pred }\OtherTok{\textless{}{-}} 
  \FunctionTok{cbind}\NormalTok{(denver\_dat\_log, allPredictions\_denver) }\SpecialCharTok{\%\textgreater{}\%} 
  \FunctionTok{mutate}\NormalTok{(}\AttributeTok{allPredictions\_denver =} \FunctionTok{round}\NormalTok{(allPredictions\_denver }\SpecialCharTok{*} \DecValTok{100}\NormalTok{)) }
\end{Highlighting}
\end{Shaded}

\begin{Shaded}
\begin{Highlighting}[]
 \FunctionTok{ggplot}\NormalTok{() }\SpecialCharTok{+} 
     \FunctionTok{geom\_sf}\NormalTok{(}\AttributeTok{data=}\NormalTok{denver\_pred, }\FunctionTok{aes}\NormalTok{(}\AttributeTok{fill=}\FunctionTok{factor}\NormalTok{(}\FunctionTok{ntile}\NormalTok{(allPredictions\_denver,}\DecValTok{4}\NormalTok{))), }
            \AttributeTok{colour=}\ConstantTok{NA}\NormalTok{) }\SpecialCharTok{+}
    \FunctionTok{scale\_fill\_manual}\NormalTok{(}\AttributeTok{values =}\NormalTok{ blues,}
                      \AttributeTok{labels=}\FunctionTok{as.character}\NormalTok{(}\FunctionTok{quantile}\NormalTok{(denver\_pred}\SpecialCharTok{$}\NormalTok{allPredictions\_denver,}
                                                 \FunctionTok{c}\NormalTok{(}\FloatTok{0.2}\NormalTok{,.}\DecValTok{4}\NormalTok{,.}\DecValTok{6}\NormalTok{,.}\DecValTok{8}\NormalTok{),}
                                                 \AttributeTok{na.rm=}\NormalTok{T)),}
                      \AttributeTok{name=}\StringTok{"Predicted}\SpecialCharTok{\textbackslash{}n}\StringTok{Probabilities(\%)}\SpecialCharTok{\textbackslash{}n}\StringTok{(Grouped in}\SpecialCharTok{\textbackslash{}n}\StringTok{Quantile}\SpecialCharTok{\textbackslash{}n}\StringTok{Breaks)"}\NormalTok{) }\SpecialCharTok{+}
\NormalTok{  mapTheme }\SpecialCharTok{+}
  \FunctionTok{labs}\NormalTok{(}\AttributeTok{title=}\StringTok{"Predicted Probability of Flood Inundation in Denver"}\NormalTok{,}
        \AttributeTok{subtitle=}\StringTok{"Based on a Logistic Regression Model trained on data from Calgary}\SpecialCharTok{\textbackslash{}n}\StringTok{"}\NormalTok{)}
\end{Highlighting}
\end{Shaded}

\includegraphics{luem_a3_markdown_presentation_files/figure-beamer/predicted_map_denver-1.pdf}

Does this line up with Denver's hydrological features?

\begin{Shaded}
\begin{Highlighting}[]
\NormalTok{denver\_hydro }\OtherTok{\textless{}{-}} \FunctionTok{read\_sf}\NormalTok{(}\StringTok{"Denver/Raw/streams/streams.shp"}\NormalTok{)}

  \FunctionTok{ggplot}\NormalTok{() }\SpecialCharTok{+} 
     \FunctionTok{geom\_sf}\NormalTok{(}\AttributeTok{data=}\NormalTok{denver\_pred, }\FunctionTok{aes}\NormalTok{(}\AttributeTok{fill=}\FunctionTok{factor}\NormalTok{(}\FunctionTok{ntile}\NormalTok{(allPredictions\_denver,}\DecValTok{4}\NormalTok{))), }
            \AttributeTok{colour=}\ConstantTok{NA}\NormalTok{) }\SpecialCharTok{+}
    \FunctionTok{scale\_fill\_manual}\NormalTok{(}\AttributeTok{values =}\NormalTok{ blues,}
                      \AttributeTok{labels=}\FunctionTok{as.character}\NormalTok{(}\FunctionTok{quantile}\NormalTok{(denver\_pred}\SpecialCharTok{$}\NormalTok{allPredictions\_denver,}
                                                 \FunctionTok{c}\NormalTok{(}\FloatTok{0.2}\NormalTok{,.}\DecValTok{4}\NormalTok{,.}\DecValTok{6}\NormalTok{,.}\DecValTok{8}\NormalTok{),}
                                                 \AttributeTok{na.rm=}\NormalTok{T)),}
                      \AttributeTok{name=}\StringTok{"Predicted}\SpecialCharTok{\textbackslash{}n}\StringTok{Probabilities(\%)}\SpecialCharTok{\textbackslash{}n}\StringTok{(Quantile}\SpecialCharTok{\textbackslash{}n}\StringTok{Breaks)"}\NormalTok{) }\SpecialCharTok{+}
   \FunctionTok{geom\_sf}\NormalTok{(}\AttributeTok{data=}\NormalTok{denver\_hydro, }\AttributeTok{color=}\StringTok{"\#EDB025"}\NormalTok{, }\AttributeTok{size=}\DecValTok{35}\NormalTok{, }\AttributeTok{linejoin=}\StringTok{"round"}\NormalTok{, }\AttributeTok{lineend=}\StringTok{"round"}\NormalTok{) }\SpecialCharTok{+}
  \FunctionTok{labs}\NormalTok{(}\AttributeTok{title=}\StringTok{"Predicted Probability of Flood Inundation in Denver"}\NormalTok{,}
        \AttributeTok{subtitle=}\StringTok{"Based on a Logistic Regression Model trained on data from Calgary}\SpecialCharTok{\textbackslash{}n}\StringTok{Yellow Lines Mark Location of Existing Rivers \& Streams"}\NormalTok{,}
        \AttributeTok{caption =} \StringTok{"Source: blah blah blah"}\NormalTok{) }\SpecialCharTok{+}
\NormalTok{   mapTheme}
\end{Highlighting}
\end{Shaded}

\includegraphics{luem_a3_markdown_presentation_files/figure-beamer/pred map 2 denver-1.pdf}

Yes, it actually lines up very well!

Phew, can't believe that worked\ldots{}
\end{block}
\end{frame}

\begin{frame}{6. Conclusion}
\protect\hypertarget{conclusion}{}
\textbf{Prediction Defense}

Our model might not be the Oracle of Delphi, but given the large number
of underlying assumptions, it works well on a number of fronts:

\begin{itemize}
\item
  It has a high level of accuracy and generalizability, as shown by its
  accuracy of 92\%, even on average across a 100-fold cross-validation
  test, and the high number of true negatives and positives compared to
  false negatives and false positives.
\item
  The predicted and observed values for Calgary line up well on the map,
  with variations observed in how each variable has been visualized -
  the former on an increasing categorical scale, and the later as a 0/1
  binary.
\item
  It accurately placed predicted inundation in Denver along the location
  of Denver's existing streams, thus confirming its high level of
  accuracy and generalizability.
\end{itemize}

This type of machine learning process differs from a site suitability
study in that it builds an algorithm based on probability and
predictions, rather than a visual assessment of existing site
conditions. This can be helpful in situations like flood inundation
mapping, when present-day data is required to communicate a range of
odds for an event that might occur in the future. In addition to
incorporating more data than a site suitability study, it can be a
useful tool for policy-makers to drive disaster-response or land
preservation programs.

It is also quicker than a more detailed, hydraulic model, to generate
spatial scenarios of relative risk, that, for a city planning department
can help prioritize how more detailed analytical resources are deployed.

\textbf{Next Steps}

If we were to run through this project again (we won't, but \emph{if} we
were), we might do a few things differently:

\begin{itemize}
\item
  Find a way to integrate land porosity through its run-off potential
  values, rather than use a simplified binary classification based on
  urbanized/non-urbanized land.
\item
  Incorporate some more variables from the stream network analysis, such
  as flow direction, that might help build the robustness of our model.
\item
  Run through the similarities and differences between the datastes for
  our two cities from the get go. We struggled for a while with a
  strange prediction of the Calgary dataset on Denver, and eventually
  learnt that: ** By not ensuring both datasets had variables in the
  same unit (either feet or metres), we were confusing the model ** By
  entering Elevation data as absolute numbers instead of in discrete
  bins, we were training the model on Calgary's relatively high range of
  elevations (900-1500 metres), which then read Denver's elevations (all
  below 900 metres), as completely inundated.
\end{itemize}

For now, we're going to get some sleep.

xx Charlie + Riddhi

\begin{block}{6.1 Endnotes}
\protect\hypertarget{endnotes}{}
\begin{enumerate}
\item
  City of Calgary. ``Flooding in Calgary - Flood of 2013.''
  www.calgary.ca. Accessed March 27, 2023.
  \url{https://www.calgary.ca/content/www/en/home/water/flooding/history-calgary.html}.
\item
  ``Flood Inundation Mapping (FIM) Program.'' n.d. U.S. Geological
  Survey. Accessed February 27, 2023.
  \url{https://www.usgs.gov/mission-areas/water-resources/science/flood-inundation-mapping-fim-program}.
\item
  Hastie, T., Tibshirani, R., Friedman, J. 2009. The Elements of
  Statistical Learning: Data Mining, Interference, and Prediction.
  Second Ed. Available at:
  \url{https://link.springer.com/book/10.1007/978-0-387-84858-7}
\end{enumerate}
\end{block}
\end{frame}

\end{document}
